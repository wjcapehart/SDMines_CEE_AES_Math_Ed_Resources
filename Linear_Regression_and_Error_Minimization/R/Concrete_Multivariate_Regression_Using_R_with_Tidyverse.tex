\documentclass[]{article}
\usepackage{lmodern}
\usepackage{amssymb,amsmath}
\usepackage{ifxetex,ifluatex}
\usepackage{fixltx2e} % provides \textsubscript
\ifnum 0\ifxetex 1\fi\ifluatex 1\fi=0 % if pdftex
  \usepackage[T1]{fontenc}
  \usepackage[utf8]{inputenc}
\else % if luatex or xelatex
  \ifxetex
    \usepackage{mathspec}
  \else
    \usepackage{fontspec}
  \fi
  \defaultfontfeatures{Ligatures=TeX,Scale=MatchLowercase}
\fi
% use upquote if available, for straight quotes in verbatim environments
\IfFileExists{upquote.sty}{\usepackage{upquote}}{}
% use microtype if available
\IfFileExists{microtype.sty}{%
\usepackage{microtype}
\UseMicrotypeSet[protrusion]{basicmath} % disable protrusion for tt fonts
}{}
\usepackage[margin=1in]{geometry}
\usepackage{hyperref}
\hypersetup{unicode=true,
            pdftitle={Visualizing Statistics and Regressions from a Spreadsheet using R},
            pdfborder={0 0 0},
            breaklinks=true}
\urlstyle{same}  % don't use monospace font for urls
\usepackage{color}
\usepackage{fancyvrb}
\newcommand{\VerbBar}{|}
\newcommand{\VERB}{\Verb[commandchars=\\\{\}]}
\DefineVerbatimEnvironment{Highlighting}{Verbatim}{commandchars=\\\{\}}
% Add ',fontsize=\small' for more characters per line
\usepackage{framed}
\definecolor{shadecolor}{RGB}{248,248,248}
\newenvironment{Shaded}{\begin{snugshade}}{\end{snugshade}}
\newcommand{\AlertTok}[1]{\textcolor[rgb]{0.94,0.16,0.16}{#1}}
\newcommand{\AnnotationTok}[1]{\textcolor[rgb]{0.56,0.35,0.01}{\textbf{\textit{#1}}}}
\newcommand{\AttributeTok}[1]{\textcolor[rgb]{0.77,0.63,0.00}{#1}}
\newcommand{\BaseNTok}[1]{\textcolor[rgb]{0.00,0.00,0.81}{#1}}
\newcommand{\BuiltInTok}[1]{#1}
\newcommand{\CharTok}[1]{\textcolor[rgb]{0.31,0.60,0.02}{#1}}
\newcommand{\CommentTok}[1]{\textcolor[rgb]{0.56,0.35,0.01}{\textit{#1}}}
\newcommand{\CommentVarTok}[1]{\textcolor[rgb]{0.56,0.35,0.01}{\textbf{\textit{#1}}}}
\newcommand{\ConstantTok}[1]{\textcolor[rgb]{0.00,0.00,0.00}{#1}}
\newcommand{\ControlFlowTok}[1]{\textcolor[rgb]{0.13,0.29,0.53}{\textbf{#1}}}
\newcommand{\DataTypeTok}[1]{\textcolor[rgb]{0.13,0.29,0.53}{#1}}
\newcommand{\DecValTok}[1]{\textcolor[rgb]{0.00,0.00,0.81}{#1}}
\newcommand{\DocumentationTok}[1]{\textcolor[rgb]{0.56,0.35,0.01}{\textbf{\textit{#1}}}}
\newcommand{\ErrorTok}[1]{\textcolor[rgb]{0.64,0.00,0.00}{\textbf{#1}}}
\newcommand{\ExtensionTok}[1]{#1}
\newcommand{\FloatTok}[1]{\textcolor[rgb]{0.00,0.00,0.81}{#1}}
\newcommand{\FunctionTok}[1]{\textcolor[rgb]{0.00,0.00,0.00}{#1}}
\newcommand{\ImportTok}[1]{#1}
\newcommand{\InformationTok}[1]{\textcolor[rgb]{0.56,0.35,0.01}{\textbf{\textit{#1}}}}
\newcommand{\KeywordTok}[1]{\textcolor[rgb]{0.13,0.29,0.53}{\textbf{#1}}}
\newcommand{\NormalTok}[1]{#1}
\newcommand{\OperatorTok}[1]{\textcolor[rgb]{0.81,0.36,0.00}{\textbf{#1}}}
\newcommand{\OtherTok}[1]{\textcolor[rgb]{0.56,0.35,0.01}{#1}}
\newcommand{\PreprocessorTok}[1]{\textcolor[rgb]{0.56,0.35,0.01}{\textit{#1}}}
\newcommand{\RegionMarkerTok}[1]{#1}
\newcommand{\SpecialCharTok}[1]{\textcolor[rgb]{0.00,0.00,0.00}{#1}}
\newcommand{\SpecialStringTok}[1]{\textcolor[rgb]{0.31,0.60,0.02}{#1}}
\newcommand{\StringTok}[1]{\textcolor[rgb]{0.31,0.60,0.02}{#1}}
\newcommand{\VariableTok}[1]{\textcolor[rgb]{0.00,0.00,0.00}{#1}}
\newcommand{\VerbatimStringTok}[1]{\textcolor[rgb]{0.31,0.60,0.02}{#1}}
\newcommand{\WarningTok}[1]{\textcolor[rgb]{0.56,0.35,0.01}{\textbf{\textit{#1}}}}
\usepackage{graphicx,grffile}
\makeatletter
\def\maxwidth{\ifdim\Gin@nat@width>\linewidth\linewidth\else\Gin@nat@width\fi}
\def\maxheight{\ifdim\Gin@nat@height>\textheight\textheight\else\Gin@nat@height\fi}
\makeatother
% Scale images if necessary, so that they will not overflow the page
% margins by default, and it is still possible to overwrite the defaults
% using explicit options in \includegraphics[width, height, ...]{}
\setkeys{Gin}{width=\maxwidth,height=\maxheight,keepaspectratio}
\IfFileExists{parskip.sty}{%
\usepackage{parskip}
}{% else
\setlength{\parindent}{0pt}
\setlength{\parskip}{6pt plus 2pt minus 1pt}
}
\setlength{\emergencystretch}{3em}  % prevent overfull lines
\providecommand{\tightlist}{%
  \setlength{\itemsep}{0pt}\setlength{\parskip}{0pt}}
\setcounter{secnumdepth}{0}
% Redefines (sub)paragraphs to behave more like sections
\ifx\paragraph\undefined\else
\let\oldparagraph\paragraph
\renewcommand{\paragraph}[1]{\oldparagraph{#1}\mbox{}}
\fi
\ifx\subparagraph\undefined\else
\let\oldsubparagraph\subparagraph
\renewcommand{\subparagraph}[1]{\oldsubparagraph{#1}\mbox{}}
\fi

%%% Use protect on footnotes to avoid problems with footnotes in titles
\let\rmarkdownfootnote\footnote%
\def\footnote{\protect\rmarkdownfootnote}

%%% Change title format to be more compact
\usepackage{titling}

% Create subtitle command for use in maketitle
\newcommand{\subtitle}[1]{
  \posttitle{
    \begin{center}\large#1\end{center}
    }
}

\setlength{\droptitle}{-2em}

  \title{Visualizing Statistics and Regressions from a Spreadsheet using R}
    \pretitle{\vspace{\droptitle}\centering\huge}
  \posttitle{\par}
    \author{}
    \preauthor{}\postauthor{}
    \date{}
    \predate{}\postdate{}
  

\begin{document}
\maketitle

This is an \href{http://rmarkdown.rstudio.com}{R Markdown} Notebook.
When you execute code within the notebook, the results appear beneath
the code.

These notebooks are typically this is designed to create a pleasing
viewing environment of data analysis that allows you to include figures,
text, links, etc. so that your work is better understood and can be
reproduced and used with confidence.

The source code for this R notebook (Rmd suffixed files), when stored as
web pages (html files), can be downloaded by clicking the button at the
top of the page.

If viewing the source code in R Studio, try executing each R ``chunk''
by clicking the \emph{Run} button within the chunk or by placing your
cursor inside it and pressing \emph{Cmd+Shift+Enter}.

\textbf{Warning. Typos are \emph{Legion}!}

\hypertarget{introduction}{%
\section{1. Introduction}\label{introduction}}

When you're in
\href{http://ecatalog.sdsmt.edu/preview_course_nopop.php?catoid=17\&coid=26571}{MATH
381 (Intro to Probability and Stats)} you'll get a taste of R. R is an
open-source statistical package build off of an earlier generation of
commercial.

The goal here is to demonstrate cracking open an excel spreadsheet in R
and calculate some basic stats, create various plots to view the
statistics, and finally, do some linear and multivariate regression

Another goal here is to show off some of R's features. R is a very
powerful tool. When translating ``powerful'' from computereese to any
frustrated human dialect, that means ``steep learning curve.'' It's also
a community-supported environment. When translating ``powerful'' from
computereese to any overscheduled human dialect, that means ``there are
LOTS of people donating packages and libraries to R.'' Some have evolved
to be a standard in the community. Others are highly specialized for a
given discipline (but have one or two items that people outside their
user communities find handy.)

But don't let that intimidate you. Once you learn one language you can
slowly pick up more. Also with this demo we aren't going to get to to be
an R guru in a day.

If you want a good stepping off point to learn R I'd recommend some of
the resources at
\href{https://www.datacamp.com/courses/free-introduction-to-r}{Data
Camp} which have some free starter tutorials for R.

\hypertarget{loading-the-libraries}{%
\section{2. Loading the Libraries}\label{loading-the-libraries}}

To work with R we will first have to load some libraries. This is like
in C where you have the \#include statement to do things like raise
things to powers and stuff like that.

Some of these libraries or ``packages'' come with R. Others will have to
be installed. Here are the ones we are using for this exercise.

Also in this exercise, we're going to use the
\href{https://www.tidyverse.org}{tidyverse} set of packages. Tidyverse
is a set of co-developed tools for data science in R. This is the new
big thing in R and is widely used so we are just going to jump in here.
SD Mines has a course beyond Engineering Stats,
\href{http://ecatalog.sdsmt.edu/preview_course_nopop.php?catoid=17\&coid=26973}{MATH
443/543 (Data Analysis)} that leverages this set of packages.

\begin{itemize}
\tightlist
\item
  Install Us First

  \begin{itemize}
  \tightlist
  \item
    \href{https://www.tidyverse.org}{tidyverse} : Set of commonly-used
    Data Science packages for R that it can install and load all at
    once. In the long-run you probably also want to install the
    tidyverse package suite anyway. For this exercise this will
    include\ldots{}

    \begin{itemize}
    \tightlist
    \item
      \href{https://ggplot2.tidyverse.org}{gplot2} : Create Elegant Data
      Visualizations Using the Grammar of Graphics
    \item
      \href{https://tidyr.tidyverse.org}{tidyr} : tools for shepherding
      data in data frames.
    \item
      \href{https://stringr.tidyverse.org}{stringr} : Simple, Consistent
      Wrappers for Common String Operations
    \end{itemize}
  \item
    \href{https://www.rdocumentation.org/packages/readxl/versions/1.1.0}{readxl}
    : also part of the \href{https://www.tidyverse.org}{tidyverse}
    package suite for reading traditional excel spreadsheets.\\
  \item
    \href{Tidyverse-Friendly\%20Introductory\%20Linear\%20Regression}{moderndive}
    : Tidyverse-Friendly Introductory Linear Regression
  \end{itemize}
\item
  This should come with R's core install, if not install 'em.

  \begin{itemize}
  \tightlist
  \item
    \href{https://www.rdocumentation.org/packages/MASS/versions/7.3-50}{MASS}
    : Has a lot of resources for regression.
  \end{itemize}
\item
  This doesn't come with R's core install so install that one\ldots{}

  \begin{itemize}
  \tightlist
  \item
    \href{https://www.rdocumentation.org/packages/moments/versions/0.14}{moments}
    : This has a load of good stuff for data analysis and plotting, more
    than you will need here, but get it anyway.
  \end{itemize}
\item
  This is a nice contributed library that lets us make pretty statistics
  tables. It was written for ecological applications but it's still
  pretty handy for looking at concrete

  \begin{itemize}
  \tightlist
  \item
    \href{https://www.rdocumentation.org/packages/pastecs/versions/1.3.21}{pastecs}:
    Package for Analysis of Space-Time Ecological Series
  \end{itemize}
\item
  Another nice contributed library that makes matrices of correlation
  coefficients look pretty (and graphically informative).

  \begin{itemize}
  \tightlist
  \item
    \href{https://www.rdocumentation.org/packages/corrplot/versions/0.84}{corrplot}
    Visualization of a Correlation Matrix
  \end{itemize}
\item
  While not officially needed for this activity but I'll demonstrate how
  units can be used in R in this example

  \begin{itemize}
  \tightlist
  \item
    \href{https://www.rdocumentation.org/packages/udunits2/versions/0.13}{udunits2}
    Provides simple bindings to Unidata's udunits library for unit
    conversions (will be demonstrating but not explicity needing it
    here)
  \item
    \href{https://www.rdocumentation.org/packages/units/versions/0.6-0}{units}
    Provides Measurement Units for R Vectors
  \end{itemize}
\end{itemize}

\begin{Shaded}
\begin{Highlighting}[]
  \CommentTok{# Tidyverse Handling Libraries}

  \KeywordTok{library}\NormalTok{(}\DataTypeTok{package =} \StringTok{"tidyverse"}\NormalTok{)  }\CommentTok{# main tidyverse suite}
\end{Highlighting}
\end{Shaded}

\begin{verbatim}
## -- Attaching packages ----------------------------------------- tidyverse 1.2.1 --
\end{verbatim}

\begin{verbatim}
## v ggplot2 3.0.0     v purrr   0.2.5
## v tibble  1.4.2     v dplyr   0.7.6
## v tidyr   0.8.1     v stringr 1.3.1
## v readr   1.1.1     v forcats 0.3.0
\end{verbatim}

\begin{verbatim}
## -- Conflicts -------------------------------------------- tidyverse_conflicts() --
## x dplyr::filter() masks stats::filter()
## x dplyr::lag()    masks stats::lag()
\end{verbatim}

\begin{Shaded}
\begin{Highlighting}[]
  \KeywordTok{library}\NormalTok{(}\DataTypeTok{package =} \StringTok{"readxl"}\NormalTok{)     }\CommentTok{# Read Excel Files}
  \KeywordTok{library}\NormalTok{(}\DataTypeTok{package =} \StringTok{"moderndive"}\NormalTok{) }\CommentTok{# regression support}

  \CommentTok{# Statistics Libraries}

  \KeywordTok{library}\NormalTok{(}\DataTypeTok{package =} \StringTok{"moments"}\NormalTok{)   }\CommentTok{# Moments, cumulants, skewness, kurtosis and related tests}
  \KeywordTok{library}\NormalTok{(}\DataTypeTok{package =} \StringTok{"MASS"}\NormalTok{)      }\CommentTok{# Support Functions and Datasets for Venables & Ripley's MASS text}
\end{Highlighting}
\end{Shaded}

\begin{verbatim}
## 
## Attaching package: 'MASS'
\end{verbatim}

\begin{verbatim}
## The following object is masked from 'package:dplyr':
## 
##     select
\end{verbatim}

\begin{Shaded}
\begin{Highlighting}[]
  \CommentTok{# Extra Graphics Libraries}

  \KeywordTok{library}\NormalTok{(}\DataTypeTok{package =} \StringTok{"corrplot"}\NormalTok{)  }\CommentTok{# Visualization of a Correlation Matrix}
\end{Highlighting}
\end{Shaded}

\begin{verbatim}
## corrplot 0.84 loaded
\end{verbatim}

\begin{Shaded}
\begin{Highlighting}[]
  \CommentTok{# Data Processing Libraries}

  \KeywordTok{library}\NormalTok{(}\DataTypeTok{package =} \StringTok{"pastecs"}\NormalTok{)   }\CommentTok{# Package for Analysis of Space-Time Ecological Series}
\end{Highlighting}
\end{Shaded}

\begin{verbatim}
## 
## Attaching package: 'pastecs'
\end{verbatim}

\begin{verbatim}
## The following objects are masked from 'package:dplyr':
## 
##     first, last
\end{verbatim}

\begin{verbatim}
## The following object is masked from 'package:tidyr':
## 
##     extract
\end{verbatim}

\begin{Shaded}
\begin{Highlighting}[]
  \KeywordTok{library}\NormalTok{(}\DataTypeTok{package =} \StringTok{"udunits2"}\NormalTok{)  }\CommentTok{# Unit Conversion Support}
\end{Highlighting}
\end{Shaded}

\begin{verbatim}
## udunits system database read
\end{verbatim}

\begin{Shaded}
\begin{Highlighting}[]
  \KeywordTok{library}\NormalTok{(}\DataTypeTok{package =} \StringTok{"units"}\NormalTok{)     }\CommentTok{# Measurement Units for R Vectors}
\end{Highlighting}
\end{Shaded}

\begin{verbatim}
## udunits system database from /usr/local/share/udunits
\end{verbatim}

\hypertarget{cracking-a-spreadsheet}{%
\section{3. Cracking a Spreadsheet}\label{cracking-a-spreadsheet}}

The spreadsheet example below is a more complicated than what you
hopefully have.

The original data set is from a set of papers on Concrete by I-Cheng Yeh

\begin{itemize}
\item
  \href{http://www.techno-press.org/content/?page=article\&journal=cac\&volume=5\&num=6\&ordernum=4}{Yeh,
  I-Cheng, ``Modeling slump of concrete with fly ash and
  superplasticizer,'' \emph{Computers and Concrete}, \textbf{5}(6),
  559-572, 2008. doi: 10.12989/cac.2008.5.6.559.}
\item
  \href{https://www.icevirtuallibrary.com/doi/10.1680/coma.2009.162.1.11}{Yeh,
  I-Cheng, ``Simulation of concrete slump using neural networks,''
  \emph{Construction Materials}, \textbf{162}(1), 11-18, 2009. doi:
  10.1680/coma.2009.162.1.11}
\item
  \href{http://www.techno-press.org/content/?page=article\&journal=cac\&volume=5\&num=1\&ordernum=1}{Yeh,
  I-Cheng, ``Prediction of workability of concrete using design of
  experiments for mixtures,'' \emph{Computers and Concrete},
  \textbf{5}(1), 1-20, 2008. doi: 10.12989/cac.2008.5.1.001}
\item
  \href{https://www.sciencedirect.com/science/article/pii/S0958946507000261?via\%3Dihub}{Yeh,
  I-Cheng, ``Modeling slump flow of concrete using second-order
  regressions and artificial neural networks,'' \emph{Cement and
  Concrete Composites}, \textbf{29}(6), 474-480, 2007. doi:
  10.1016/j.cemconcomp.2007.02.001}
\item
  \href{https://ascelibrary.org/doi/10.1061/\%28ASCE\%290887-3801\%282006\%2920\%3A3\%28217\%29}{Yeh,
  I-Cheng, ``Exploring concrete slump model using artificial neural
  networks,'' \emph{ASCE J. of Computing in Civil Engineering},
  \textbf{20}(3), 217-221, 2006. doi:
  10.1061/(ASCE)0887-3801(2006)20:3(217)}
\end{itemize}

and is kept at the
\href{https://archive.ics.uci.edu/ml/datasets/Concrete+Slump+Test}{UC-Irvine
Machine Learning Repository}.

It can be found here at
\url{http://kyrill.ias.sdsmt.edu/cee_284/Base_Concrete_Slump_Test_for_R.xlsx}

The relevant page and screenshot is below. For drama-free R import you
are probably best off keeping a page on your spreadsheet file that is
very simple, with numbers going down, and a single line for Row-1 with
the headers of each column. If you want to get fancy on other pages that
you'd turn in as tables in reports, you can do that on another
spreadsheet page.

\begin{figure}
\centering
\includegraphics{http://kyrill.ias.sdsmt.edu/cee_284/Base_Concrete_Slump_Test_for_R.png}
\caption{Concrete Spreadsheet Screenshot}
\end{figure}

To crack open the spreadsheet we will want to use the
\href{https://www.rdocumentation.org/packages/readxl/versions/1.1.0/topics/read_excel}{read\_excel}
function.

You can read the spreadsheet from a local drive or from a website.

\begin{Shaded}
\begin{Highlighting}[]
  \CommentTok{# you will need the full path to the file you are using (either online or locally on your disk)}

  \CommentTok{# The if else block should query your machine to determine which operating system.}
  \CommentTok{#  if you are not bi-platform, you likely don't need this.}

  \ControlFlowTok{if}\NormalTok{(.Platform}\OperatorTok{$}\NormalTok{OS.type }\OperatorTok{==}\StringTok{ "windows"}\NormalTok{) \{}
    \CommentTok{# Windows}
\NormalTok{    spreadsheet_name     =}\StringTok{ "%HOMEPATH%/Downloads/Base_Concrete_Slump_Test_for_R.xlsx"}
\NormalTok{  \} }\ControlFlowTok{else}\NormalTok{ \{}
    \CommentTok{# Unix (Linux, MacOS, Solaris)}
\NormalTok{    spreadsheet_name     =}\StringTok{ "~/Downloads/Base_Concrete_Slump_Test_for_R.xlsx"}
\NormalTok{  \}}


  \CommentTok{# I am keeping a copy of these spreadsheet at the URL below.  It can be downloaded automatically}
  \CommentTok{#   and then loaded.  We can also discretely delete it when done.}

\NormalTok{      spreadsheet_url =}\StringTok{ "http://kyrill.ias.sdsmt.edu/cee_284/Base_Concrete_Slump_Test_for_R.xlsx"}
    
      \KeywordTok{download.file}\NormalTok{(}\DataTypeTok{url      =}\NormalTok{   spreadsheet_url, }\CommentTok{# URL location}
                    \DataTypeTok{destfile =}\NormalTok{ spreadsheet_name) }\CommentTok{# local downloaded location}
      
      \KeywordTok{remove}\NormalTok{(spreadsheet_url) }\CommentTok{# clean up variables}
  
  \CommentTok{# this command will read the file}

\NormalTok{  concrete =}\StringTok{ }\KeywordTok{read_excel}\NormalTok{(}\DataTypeTok{path      =}\NormalTok{ spreadsheet_name, }\CommentTok{# local spreadsheet location}
                        \DataTypeTok{sheet     =}           \StringTok{"Data"}\NormalTok{, }\CommentTok{# page of spreadsheet}
                        \DataTypeTok{col_names =}             \OtherTok{TRUE}\NormalTok{) }\CommentTok{# first row are the column headers}
  
  
  \CommentTok{# clean up your hard drive!  Don't be like me!}

  \ControlFlowTok{if}\NormalTok{(.Platform}\OperatorTok{$}\NormalTok{OS.type }\OperatorTok{==}\StringTok{ "windows"}\NormalTok{) \{}
    \CommentTok{# Windows}
    \KeywordTok{system}\NormalTok{(}\KeywordTok{str_c}\NormalTok{(}\StringTok{"DEL   "}\NormalTok{, }
\NormalTok{                 spreadsheet_name,}
                 \DataTypeTok{sep=}\StringTok{""}\NormalTok{))}
\NormalTok{    \} }\ControlFlowTok{else}\NormalTok{ \{}
    \CommentTok{# Unix (Linux, MacOS, Solaris)}
    \KeywordTok{system}\NormalTok{(}\KeywordTok{str_c}\NormalTok{(}\StringTok{"rm -v  "}\NormalTok{, }
\NormalTok{                 spreadsheet_name,}
                 \DataTypeTok{sep=}\StringTok{""}\NormalTok{))}
\NormalTok{      \}}
  
  \KeywordTok{remove}\NormalTok{(spreadsheet_name) }\CommentTok{# clean up variables}
\end{Highlighting}
\end{Shaded}

With the data read in we can now look at the table of the data. This
looks much nicer when working in R Notebooks instead of Plain Ordinary
R.

\begin{Shaded}
\begin{Highlighting}[]
  \CommentTok{# Print data frame}
  \KeywordTok{colnames}\NormalTok{(concrete)[}\DecValTok{1}\NormalTok{] =}\StringTok{ "Test_Number"}
  \KeywordTok{print}\NormalTok{(concrete)}
\end{Highlighting}
\end{Shaded}

\begin{verbatim}
## # A tibble: 103 x 11
##    Test_Number Cement  Slag Fly_Ash Water Superplasticizer Coarse_Aggregat~
##          <dbl>  <dbl> <dbl>   <dbl> <dbl>            <dbl>            <dbl>
##  1           1    273    82     105   210                9              904
##  2           2    163   149     191   180               12              843
##  3           3    162   148     191   179               16              840
##  4           4    162   148     190   179               19              838
##  5           5    154   112     144   220               10              923
##  6           6    147    89     115   202                9              860
##  7           7    152   139     178   168               18              944
##  8           8    145     0     227   240                6              750
##  9           9    152     0     237   204                6              785
## 10          10    304     0     140   214                6              895
## # ... with 93 more rows, and 4 more variables: Fine_Aggregates <dbl>,
## #   Slump <dbl>, Flow <dbl>, Compressive_Strength_28dy <dbl>
\end{verbatim}

\hypertarget{extra-units-not-part-of-this-exercise-but-its-a-nifty-tangent}{%
\subsubsection{Extra: Units (not part of this exercise but it's a nifty
tangent)}\label{extra-units-not-part-of-this-exercise-but-its-a-nifty-tangent}}

*Dang. I like units. I don't see any. I'm anal and have learned that
adding as much descriptive data early on in processing your data set
will make people (and most importantly, yourself) not hate you at a
later date. So I am adding them here with the
\href{https://www.rdocumentation.org/packages/units/versions/0.6-0/topics/set_units}{set\_units}
function. This will add units as an attribute.

Units don't work with everything and you should probably keep a copy of
your original un-unitted data frame.

\begin{Shaded}
\begin{Highlighting}[]
\CommentTok{# first we clone our data frame}

\NormalTok{concrete_units =}\StringTok{ }\NormalTok{concrete}

\NormalTok{concrete_units}\OperatorTok{$}\NormalTok{Cement                    =}\StringTok{ }\KeywordTok{set_units}\NormalTok{(}\DataTypeTok{x     =}\NormalTok{ concrete_units}\OperatorTok{$}\NormalTok{Cement, }
                                                     \DataTypeTok{value =} \StringTok{"kg m-3"}\NormalTok{)}

\NormalTok{concrete_units}\OperatorTok{$}\NormalTok{Slag                      =}\StringTok{ }\KeywordTok{set_units}\NormalTok{(}\DataTypeTok{x     =}\NormalTok{ concrete_units}\OperatorTok{$}\NormalTok{Slag, }
                                                     \DataTypeTok{value =} \StringTok{"kg m-3"}\NormalTok{)}

\NormalTok{concrete_units}\OperatorTok{$}\NormalTok{Fly_Ash                   =}\StringTok{ }\KeywordTok{set_units}\NormalTok{(}\DataTypeTok{x     =}\NormalTok{ concrete_units}\OperatorTok{$}\NormalTok{Fly_Ash, }
                                                     \DataTypeTok{value =} \StringTok{"kg m-3"}\NormalTok{)}

\NormalTok{concrete_units}\OperatorTok{$}\NormalTok{Water                     =}\StringTok{ }\KeywordTok{set_units}\NormalTok{(}\DataTypeTok{x     =}\NormalTok{ concrete_units}\OperatorTok{$}\NormalTok{Water, }
                                                     \DataTypeTok{value =} \StringTok{"kg m-3"}\NormalTok{)}

\NormalTok{concrete_units}\OperatorTok{$}\NormalTok{Superplasticizer          =}\StringTok{ }\KeywordTok{set_units}\NormalTok{(}\DataTypeTok{x     =}\NormalTok{ concrete_units}\OperatorTok{$}\NormalTok{Superplasticizer, }
                                                     \DataTypeTok{value =} \StringTok{"kg m-3"}\NormalTok{)}

\NormalTok{concrete_units}\OperatorTok{$}\NormalTok{Coarse_Aggregates         =}\StringTok{ }\KeywordTok{set_units}\NormalTok{(}\DataTypeTok{x     =}\NormalTok{ concrete_units}\OperatorTok{$}\NormalTok{Coarse_Aggregates, }
                                                     \DataTypeTok{value =} \StringTok{"kg m-3"}\NormalTok{)}

\NormalTok{concrete_units}\OperatorTok{$}\NormalTok{Fine_Aggregates           =}\StringTok{ }\KeywordTok{set_units}\NormalTok{(}\DataTypeTok{x     =}\NormalTok{ concrete_units}\OperatorTok{$}\NormalTok{Fine_Aggregates, }
                                                     \DataTypeTok{value =} \StringTok{"kg m-3"}\NormalTok{)}

\NormalTok{concrete_units}\OperatorTok{$}\NormalTok{Slump                     =}\StringTok{ }\KeywordTok{set_units}\NormalTok{(}\DataTypeTok{x     =}\NormalTok{ concrete_units}\OperatorTok{$}\NormalTok{Slump, }
                                                     \DataTypeTok{value =} \StringTok{"cm"}\NormalTok{)}

\NormalTok{concrete_units}\OperatorTok{$}\NormalTok{Flow                      =}\StringTok{ }\KeywordTok{set_units}\NormalTok{(}\DataTypeTok{x     =}\NormalTok{ concrete_units}\OperatorTok{$}\NormalTok{Flow, }
                                                     \DataTypeTok{value =} \StringTok{"cm"}\NormalTok{)}

\NormalTok{concrete_units}\OperatorTok{$}\NormalTok{Compressive_Strength_28dy =}\StringTok{ }\KeywordTok{set_units}\NormalTok{(}\DataTypeTok{x     =}\NormalTok{ concrete_units}\OperatorTok{$}\NormalTok{Compressive_Strength_28dy, }
                                                     \DataTypeTok{value =} \StringTok{"MPa"}\NormalTok{)}

\KeywordTok{print}\NormalTok{(concrete_units)}
\end{Highlighting}
\end{Shaded}

\begin{verbatim}
## # A tibble: 103 x 11
##    Test_Number   Cement     Slag  Fly_Ash    Water Superplasticizer
##          <dbl> [kg/m^3] [kg/m^3] [kg/m^3] [kg/m^3]         [kg/m^3]
##  1           1      273       82      105      210                9
##  2           2      163      149      191      180               12
##  3           3      162      148      191      179               16
##  4           4      162      148      190      179               19
##  5           5      154      112      144      220               10
##  6           6      147       89      115      202                9
##  7           7      152      139      178      168               18
##  8           8      145        0      227      240                6
##  9           9      152        0      237      204                6
## 10          10      304        0      140      214                6
## # ... with 93 more rows, and 5 more variables: Coarse_Aggregates [kg/m^3],
## #   Fine_Aggregates [kg/m^3], Slump [cm], Flow [cm],
## #   Compressive_Strength_28dy [MPa]
\end{verbatim}

If you click in the Global Environment Box, those units aren't arbitrary
strings. They are listed as numerators, denominators and also the way in
which squares, etc., are archived are explicit.

Better Still, the same command of set\_units when applied to a variable
that already has units will convert it. This is nice when moving between
SI units, USCS units. {[}If you are going to be cheeky and try the
Furlong/Firkin/Fortnight system (FFF), sorry to disappoint, that while
the udunits2 package in R recognizes all three units, it recognizes
firkins as a volume measure (which is really is) and not the mass
measure based on density of water.{]}

Example here:

\begin{Shaded}
\begin{Highlighting}[]
  \CommentTok{# a little unit-fu™️ play!}

\NormalTok{  strength_in_psi =}\StringTok{ }\KeywordTok{set_units}\NormalTok{(}\DataTypeTok{x     =}\NormalTok{ concrete_units}\OperatorTok{$}\NormalTok{Compressive_Strength_28dy,}
                              \DataTypeTok{value =} \StringTok{"psi"}\NormalTok{)}

  \KeywordTok{print}\NormalTok{(concrete_units}\OperatorTok{$}\NormalTok{Compressive_Strength_28dy[}\DecValTok{1}\NormalTok{])}
\end{Highlighting}
\end{Shaded}

\begin{verbatim}
## 34.99 [MPa]
\end{verbatim}

\begin{Shaded}
\begin{Highlighting}[]
  \KeywordTok{print}\NormalTok{(strength_in_psi[}\DecValTok{1}\NormalTok{])}
\end{Highlighting}
\end{Shaded}

\begin{verbatim}
## 5074.87 [psi]
\end{verbatim}

\begin{Shaded}
\begin{Highlighting}[]
  \CommentTok{# Ok now I'm being silly but so were the package developers.  }
  \CommentTok{# Blame them.  }
  \CommentTok{# (Once again, I can't do official FFF units)}

\NormalTok{  cement_in_slug_per_cu3 =}\StringTok{ }\KeywordTok{set_units}\NormalTok{(}\DataTypeTok{x     =}\NormalTok{ concrete_units}\OperatorTok{$}\NormalTok{Cement,}
                                     \DataTypeTok{value =} \StringTok{"slugs/furlongs^3"}\NormalTok{)}
  
  \KeywordTok{print}\NormalTok{(concrete_units}\OperatorTok{$}\NormalTok{Cement[}\DecValTok{1}\NormalTok{])}
\end{Highlighting}
\end{Shaded}

\begin{verbatim}
## 273 [kg/m^3]
\end{verbatim}

\begin{Shaded}
\begin{Highlighting}[]
  \KeywordTok{print}\NormalTok{(cement_in_slug_per_cu3[}\DecValTok{1}\NormalTok{])}
\end{Highlighting}
\end{Shaded}

\begin{verbatim}
## 152289718 [slugs/furlongs^3]
\end{verbatim}

\begin{Shaded}
\begin{Highlighting}[]
  \CommentTok{# cleaning-up our horseplay..}
  
  \KeywordTok{remove}\NormalTok{(strength_in_psi)}
  \KeywordTok{remove}\NormalTok{(cement_in_slug_per_cu3)}
  
  \KeywordTok{remove}\NormalTok{(concrete_units)}
\end{Highlighting}
\end{Shaded}

Caveat! As useful as this can be, know this: Not all R functions play
nice with units or other ``attributes'' in data frames Some of the
plotting routines and linear regression routines below will work with
this.

If you need your units and want to minimize ``messy'' code in R when it
conflicts any given function. You can later strip out units by using the
\href{https://www.rdocumentation.org/packages/base/versions/3.5.1/topics/numeric}{as.numeric()}
function

\hypertarget{some-basic-statistics-and-traditional-single-variable-plots}{%
\section{4. Some Basic Statistics and Traditional Single Variable
Plots}\label{some-basic-statistics-and-traditional-single-variable-plots}}

Lets start with some basic statistics and plotting of them.

\hypertarget{the-classic-stats}{%
\subsection{4.1. The ``classic'' stats}\label{the-classic-stats}}

Let's get the mom-and-apple-pie stats for Concrete That second argument
allows you to deal with missing data.

\begin{Shaded}
\begin{Highlighting}[]
  \CommentTok{# statistics for cement}


  \KeywordTok{print}\NormalTok{(}\KeywordTok{str_c}\NormalTok{(}\StringTok{"    Mean Cement : "}\NormalTok{,}
              \KeywordTok{mean}\NormalTok{(}\DataTypeTok{x     =}\NormalTok{ concrete}\OperatorTok{$}\NormalTok{Cement, }\CommentTok{# variable to crunch}
                   \DataTypeTok{na.rm =}            \OtherTok{TRUE}\NormalTok{) }\CommentTok{# ignore msissing data}
\NormalTok{              ))}
\end{Highlighting}
\end{Shaded}

\begin{verbatim}
## [1] "    Mean Cement : 229.894174757282"
\end{verbatim}

\begin{Shaded}
\begin{Highlighting}[]
  \KeywordTok{print}\NormalTok{(}\KeywordTok{str_c}\NormalTok{(}\StringTok{"   Stdev Cement : "}\NormalTok{,}
              \KeywordTok{sd}\NormalTok{(}\DataTypeTok{x     =}\NormalTok{ concrete}\OperatorTok{$}\NormalTok{Cement, }\CommentTok{# variable to crunch}
                 \DataTypeTok{na.rm =}            \OtherTok{TRUE}\NormalTok{) }\CommentTok{# ignore msissing data}
\NormalTok{              ))}
\end{Highlighting}
\end{Shaded}

\begin{verbatim}
## [1] "   Stdev Cement : 78.8772300268858"
\end{verbatim}

\begin{Shaded}
\begin{Highlighting}[]
  \KeywordTok{print}\NormalTok{(}\KeywordTok{str_c}\NormalTok{(}\StringTok{"Skewness Cement : "}\NormalTok{,}
              \KeywordTok{skewness}\NormalTok{(}\DataTypeTok{x     =}\NormalTok{ concrete}\OperatorTok{$}\NormalTok{Cement, }\CommentTok{# variable to crunch}
                       \DataTypeTok{na.rm =}            \OtherTok{TRUE}\NormalTok{) }\CommentTok{# ignore msissing data}
\NormalTok{              ))}
\end{Highlighting}
\end{Shaded}

\begin{verbatim}
## [1] "Skewness Cement : 0.143018080025135"
\end{verbatim}

\begin{Shaded}
\begin{Highlighting}[]
  \KeywordTok{print}\NormalTok{(}\KeywordTok{str_c}\NormalTok{(}\StringTok{"Kurtosis Cement : "}\NormalTok{,}
              \KeywordTok{kurtosis}\NormalTok{(}\DataTypeTok{x     =}\NormalTok{ concrete}\OperatorTok{$}\NormalTok{Cement, }\CommentTok{# variable to crunch}
                       \DataTypeTok{na.rm =}            \OtherTok{TRUE}\NormalTok{) }\CommentTok{# ignore msissing data}
\NormalTok{              ))}
\end{Highlighting}
\end{Shaded}

\begin{verbatim}
## [1] "Kurtosis Cement : 1.33448397363582"
\end{verbatim}

OK this is a little clunky. It would be nice if someone somewhere made a
support library for R that will make nice tables of statistics.

In this case Vive La France! A team from French Research Institute for
Exploitation of the Sea thought the same question and as is often the
case for the R community not only drafted a set of tools to do this,
\emph{and} made it public.

Here we ware using their
\href{https://www.rdocumentation.org/packages/pastecs/versions/1.3.21/topics/stat.desc}{stat.desc}
function.

This will hopefully give people wanting to make basic tables ``maximum
satisfaction with minimal effort.''

\begin{Shaded}
\begin{Highlighting}[]
  \CommentTok{# Plot a statistics table -- all the classics nice and handy and pretty.}

  \KeywordTok{options}\NormalTok{(}\DataTypeTok{digits=}\DecValTok{2}\NormalTok{) }\CommentTok{# this simply set the decimal count in the table to be created below  }
                    \CommentTok{# this particular function creates the table in scientific notation}
  
\NormalTok{  concrete_statistics =}\StringTok{ }\KeywordTok{stat.desc}\NormalTok{(}\DataTypeTok{x    =}\NormalTok{ concrete,  }\CommentTok{# data frame}
                                  \DataTypeTok{basic =}    \OtherTok{TRUE}\NormalTok{,  }\CommentTok{# includes counts and extremes }
                                  \DataTypeTok{desc =}     \OtherTok{TRUE}\NormalTok{,  }\CommentTok{# include classic stats (mean etc)}
                                  \DataTypeTok{norm =}     \OtherTok{TRUE}\NormalTok{,  }\CommentTok{# include normal dist stats (skewness etc)}
                                  \DataTypeTok{p    =}     \FloatTok{0.95}\NormalTok{)  }\CommentTok{# use 95% confidence limits}


  \KeywordTok{print}\NormalTok{(concrete_statistics)}
\end{Highlighting}
\end{Shaded}

\begin{verbatim}
##              Test_Number   Cement     Slag  Fly_Ash    Water
## nbr.val          1.0e+02  1.0e+02  1.0e+02  1.0e+02  1.0e+02
## nbr.null         0.0e+00  0.0e+00  2.6e+01  2.0e+01  0.0e+00
## nbr.na           0.0e+00  0.0e+00  0.0e+00  0.0e+00  0.0e+00
## min              1.0e+00  1.4e+02  0.0e+00  0.0e+00  1.6e+02
## max              1.0e+02  3.7e+02  1.9e+02  2.6e+02  2.4e+02
## range            1.0e+02  2.4e+02  1.9e+02  2.6e+02  8.0e+01
## sum              5.4e+03  2.4e+04  8.0e+03  1.5e+04  2.0e+04
## median           5.2e+01  2.5e+02  1.0e+02  1.6e+02  2.0e+02
## mean             5.2e+01  2.3e+02  7.8e+01  1.5e+02  2.0e+02
## SE.mean          2.9e+00  7.8e+00  6.0e+00  8.4e+00  2.0e+00
## CI.mean.0.95     5.8e+00  1.5e+01  1.2e+01  1.7e+01  3.9e+00
## var              8.9e+02  6.2e+03  3.7e+03  7.3e+03  4.1e+02
## std.dev          3.0e+01  7.9e+01  6.0e+01  8.5e+01  2.0e+01
## coef.var         5.7e-01  3.4e-01  7.8e-01  5.7e-01  1.0e-01
## skewness         0.0e+00  1.4e-01 -1.9e-01 -6.6e-01  2.6e-01
## skew.2SE         0.0e+00  3.0e-01 -3.9e-01 -1.4e+00  5.4e-01
## kurtosis        -1.2e+00 -1.7e+00 -1.4e+00 -8.0e-01 -8.5e-01
## kurt.2SE        -1.3e+00 -1.8e+00 -1.5e+00 -8.5e-01 -9.1e-01
## normtest.W       9.5e-01  8.4e-01  8.6e-01  8.6e-01  9.7e-01
## normtest.p       1.4e-03  2.9e-09  2.1e-08  1.6e-08  1.2e-02
##              Superplasticizer Coarse_Aggregates Fine_Aggregates    Slump
## nbr.val               1.0e+02           1.0e+02         1.0e+02  1.0e+02
## nbr.null              0.0e+00           0.0e+00         0.0e+00  1.1e+01
## nbr.na                0.0e+00           0.0e+00         0.0e+00  0.0e+00
## min                   4.4e+00           7.1e+02         6.4e+02  0.0e+00
## max                   1.9e+01           1.0e+03         9.0e+02  2.9e+01
## range                 1.5e+01           3.4e+02         2.6e+02  2.9e+01
## sum                   8.8e+02           9.1e+04         7.6e+04  1.9e+03
## median                8.0e+00           8.8e+02         7.4e+02  2.2e+01
## mean                  8.5e+00           8.8e+02         7.4e+02  1.8e+01
## SE.mean               2.8e-01           8.7e+00         6.2e+00  8.6e-01
## CI.mean.0.95          5.5e-01           1.7e+01         1.2e+01  1.7e+00
## var                   7.9e+00           7.8e+03         4.0e+03  7.7e+01
## std.dev               2.8e+00           8.8e+01         6.3e+01  8.8e+00
## coef.var              3.3e-01           1.0e-01         8.6e-02  4.8e-01
## skewness              1.1e+00           1.2e-01         2.6e-01 -1.1e+00
## skew.2SE              2.3e+00           2.5e-01         5.4e-01 -2.3e+00
## kurtosis              1.6e+00          -8.8e-01        -6.9e-01 -2.0e-01
## kurt.2SE              1.7e+00          -9.3e-01        -7.3e-01 -2.1e-01
## normtest.W            9.0e-01           9.7e-01         9.7e-01  8.1e-01
## normtest.p            1.4e-06           2.9e-02         1.5e-02  4.4e-10
##                  Flow Compressive_Strength_28dy
## nbr.val       1.0e+02                   1.0e+02
## nbr.null      0.0e+00                   0.0e+00
## nbr.na        0.0e+00                   0.0e+00
## min           2.0e+01                   1.7e+01
## max           7.8e+01                   5.9e+01
## range         5.8e+01                   4.1e+01
## sum           5.1e+03                   3.7e+03
## median        5.4e+01                   3.6e+01
## mean          5.0e+01                   3.6e+01
## SE.mean       1.7e+00                   7.7e-01
## CI.mean.0.95  3.4e+00                   1.5e+00
## var           3.1e+02                   6.1e+01
## std.dev       1.8e+01                   7.8e+00
## coef.var      3.5e-01                   2.2e-01
## skewness     -5.1e-01                   1.9e-01
## skew.2SE     -1.1e+00                   3.9e-01
## kurtosis     -9.5e-01                   7.5e-02
## kurt.2SE     -1.0e+00                   7.9e-02
## normtest.W    9.1e-01                   9.9e-01
## normtest.p    2.0e-06                   4.8e-01
\end{verbatim}

\hypertarget{reorganizing-your-data-to-handle-multiple-variables-at-once}{%
\subsection{4.2. Reorganizing Your Data to Handle Multiple Variables at
Once}\label{reorganizing-your-data-to-handle-multiple-variables-at-once}}

To leverage some of R's more nifty features we will need to reorganize
our data from a ``spreadsheet style'' format to what some people have
called a ``long form'' table so that the column headers of our concrete
traits become a single column with the values in the columns placed all
into a single column similar to the graphic below.

\begin{figure}
\centering
\includegraphics{https://jules32.github.io/2016-07-12-Oxford/dplyr_tidyr/img/rstudio-cheatsheet-reshaping-data-gather.png}
\caption{Example of the Gather Function}
\end{figure}

This is done with the function
\href{https://www.rdocumentation.org/packages/tidyr/versions/0.8.1/topics/gather}{gather()}

\begin{Shaded}
\begin{Highlighting}[]
  \CommentTok{# Gathering our components into a single column.}

  \CommentTok{# We just want the names of our components here so we get everything past}
  \CommentTok{# the first column (which is the experiment name)}

\NormalTok{  column_names  =}\StringTok{ }\KeywordTok{colnames}\NormalTok{(concrete[}\DecValTok{2}\OperatorTok{:}\KeywordTok{ncol}\NormalTok{(concrete)])   }

  \KeywordTok{tbl_df}\NormalTok{(column_names) }\CommentTok{# tbl_df makes it look pretty when printed}
\end{Highlighting}
\end{Shaded}

\begin{verbatim}
## # A tibble: 10 x 1
##    value                    
##    <chr>                    
##  1 Cement                   
##  2 Slag                     
##  3 Fly_Ash                  
##  4 Water                    
##  5 Superplasticizer         
##  6 Coarse_Aggregates        
##  7 Fine_Aggregates          
##  8 Slump                    
##  9 Flow                     
## 10 Compressive_Strength_28dy
\end{verbatim}

\begin{Shaded}
\begin{Highlighting}[]
  \CommentTok{# the gather command will group everything. in the column name group }

\NormalTok{  concrete_tidy =}\StringTok{ }\KeywordTok{gather}\NormalTok{(}\DataTypeTok{data  =}\NormalTok{    concrete, }\CommentTok{# your data frame}
                         \DataTypeTok{key   =} \StringTok{"Parameter"}\NormalTok{, }\CommentTok{# column name for your former columns}
                         \DataTypeTok{value =}     \StringTok{"Value"}\NormalTok{, }\CommentTok{# column name for your data}
\NormalTok{                         column_names       ) }\CommentTok{# the list for the columns to "gather"}

  

  \CommentTok{# this will let us sort future plots in the same order as our plots.  }
  
\NormalTok{  concrete_tidy}\OperatorTok{$}\NormalTok{Parameter =}\StringTok{ }\KeywordTok{factor}\NormalTok{(}\DataTypeTok{x      =}\NormalTok{ concrete_tidy}\OperatorTok{$}\NormalTok{Parameter,}
                                   \DataTypeTok{levels =}\NormalTok{ column_names)}
  
  \CommentTok{# we can also split things between our dependant variables and independant variables.}
  
  
\NormalTok{  concrete_independent =}\StringTok{ }\KeywordTok{subset}\NormalTok{(}\DataTypeTok{x      =}\NormalTok{ concrete_tidy,}
                                \DataTypeTok{subset =}\NormalTok{ (Parameter }\OperatorTok{!=}\StringTok{ "Slump"}\NormalTok{) }\OperatorTok{&}
\StringTok{                                         }\NormalTok{(Parameter }\OperatorTok{!=}\StringTok{ "Flow"}\NormalTok{)  }\OperatorTok{&}
\StringTok{                                         }\NormalTok{(Parameter }\OperatorTok{!=}\StringTok{ "Compressive_Strength_28dy"}\NormalTok{)}
\NormalTok{                                ) }
    
    
\NormalTok{  concrete_dependent =}\StringTok{ }\KeywordTok{subset}\NormalTok{(}\DataTypeTok{x      =}\NormalTok{ concrete_tidy,}
                              \DataTypeTok{subset =}\NormalTok{ (Parameter }\OperatorTok{==}\StringTok{ "Slump"}\NormalTok{) }\OperatorTok{|}
\StringTok{                                       }\NormalTok{(Parameter }\OperatorTok{==}\StringTok{ "Flow"}\NormalTok{)  }\OperatorTok{|}
\StringTok{                                       }\NormalTok{(Parameter }\OperatorTok{==}\StringTok{ "Compressive_Strength_28dy"}\NormalTok{)}
\NormalTok{                              )}

 
                       

  \KeywordTok{print}\NormalTok{(concrete_tidy)}
\end{Highlighting}
\end{Shaded}

\begin{verbatim}
## # A tibble: 1,030 x 3
##    Test_Number Parameter Value
##          <dbl> <fct>     <dbl>
##  1           1 Cement      273
##  2           2 Cement      163
##  3           3 Cement      162
##  4           4 Cement      162
##  5           5 Cement      154
##  6           6 Cement      147
##  7           7 Cement      152
##  8           8 Cement      145
##  9           9 Cement      152
## 10          10 Cement      304
## # ... with 1,020 more rows
\end{verbatim}

\begin{Shaded}
\begin{Highlighting}[]
  \KeywordTok{print}\NormalTok{(concrete_independent)}
\end{Highlighting}
\end{Shaded}

\begin{verbatim}
## # A tibble: 721 x 3
##    Test_Number Parameter Value
##          <dbl> <fct>     <dbl>
##  1           1 Cement      273
##  2           2 Cement      163
##  3           3 Cement      162
##  4           4 Cement      162
##  5           5 Cement      154
##  6           6 Cement      147
##  7           7 Cement      152
##  8           8 Cement      145
##  9           9 Cement      152
## 10          10 Cement      304
## # ... with 711 more rows
\end{verbatim}

\begin{Shaded}
\begin{Highlighting}[]
  \KeywordTok{print}\NormalTok{(concrete_dependent)}
\end{Highlighting}
\end{Shaded}

\begin{verbatim}
## # A tibble: 309 x 3
##    Test_Number Parameter Value
##          <dbl> <fct>     <dbl>
##  1           1 Slump      23  
##  2           2 Slump       0  
##  3           3 Slump       1  
##  4           4 Slump       3  
##  5           5 Slump      20  
##  6           6 Slump      23  
##  7           7 Slump       0  
##  8           8 Slump      14.5
##  9           9 Slump      15.5
## 10          10 Slump      19  
## # ... with 299 more rows
\end{verbatim}

\hypertarget{plotting-graphics-using-tidyverse-resources}{%
\section{5. Plotting Graphics using Tidyverse
Resources}\label{plotting-graphics-using-tidyverse-resources}}

R has a few ways to do the basic histograms, Boxplots and other
distribution plots.

There are a number of spiffy ways to plot these statistical plots in R.
We're just using one here\ldots{}

\hypertarget{sloooowwwwlllllyyy-making-a-simple-plot-histogram-edition}{%
\subsection{5.1. SLOOOOWWWWLLLLLYYY Making a Simple Plot (Histogram
Edition)}\label{sloooowwwwlllllyyy-making-a-simple-plot-histogram-edition}}

Now I'm going to do this one tiny step at a time until we get to a
viable product. (This is how I work through cryptic procedures so I can
see what each little additional mystery thingie does.)

Graphing is invoked by the \href{https://ggplot2.tidyverse.org}{ggplot2}
command.. which has a heluvalot under its hood! For me all that detail
was what had me a little shy to adopt this way of printing data.

Tidyverse uses what is sometimes called the
\href{https://ramnathv.github.io/pycon2014-r/visualize/ggplot2.html}{``grammar
of graphics''} method\ldots{} to make a long story longer, the GoG
presents separate commands to do separate things rather bundle stuff in
a single graphing function. Sometimes it makes a lot of sense\ldots{}
other times it may be confusion. (Hence me demonstrating making a graph
this one tiny step at a time!

First thing we are going to do is open a plotting space with the command
\href{https://ggplot2.tidyverse.org/reference/ggplot.html}{ggplot()}

\begin{Shaded}
\begin{Highlighting}[]
\CommentTok{# invoke the ggplot plotting environmnent.}

\KeywordTok{ggplot}\NormalTok{() }
\end{Highlighting}
\end{Shaded}

\includegraphics{Concrete_Multivariate_Regression_Using_R_with_Tidyverse_files/figure-latex/unnamed-chunk-9-1.pdf}

Wow. We have a\ldots{} big square of\ldots{} grey. All it's doing is
setting up our plot environment\ldots{} so let's do some more\ldots{}

If we want to do a histogram we are going to have to tell it what we
want to print and where to get the stuff

When we add things to a plot command in Tidyverse we ``add'' to the
steps incrementally.

This involves a ``mapping'' function called
``\href{https://ggplot2.tidyverse.org/reference/aes.html}{aes}'' (short
for aesthetics)

here, we are working with the data frame ``concrete'' and are working on
the variable Cement which we are tossing onto the x axis because that's
where the bins of cement go!

\begin{Shaded}
\begin{Highlighting}[]
\KeywordTok{ggplot}\NormalTok{(}\DataTypeTok{data =}\NormalTok{ concrete) }\OperatorTok{+}\StringTok{   }\CommentTok{# EDIT:  invoke graphics environment using a given dataframe}
\StringTok{  }
\StringTok{  }\KeywordTok{aes}\NormalTok{(}\DataTypeTok{x    =}\NormalTok{ Cement)        }\CommentTok{# NEW: select variable to print... You can get really fancy here later}
\end{Highlighting}
\end{Shaded}

\includegraphics{Concrete_Multivariate_Regression_Using_R_with_Tidyverse_files/figure-latex/unnamed-chunk-10-1.pdf}

OK now we have something that looks like we may have the making of the
graph. If you don't like grey outlines and white grids, no worries, we
can change that shortly.

OK.. we are now ready to make a histogram\ldots{}

Here we will use one of the gglot2's "geom\_*" (draw stuff) resources.
The default should work for us here.

\begin{Shaded}
\begin{Highlighting}[]
\KeywordTok{ggplot}\NormalTok{(}\DataTypeTok{data =}\NormalTok{ concrete) }\OperatorTok{+}\StringTok{   }\CommentTok{# invoke graphics environment using a given dataframe}
\StringTok{  }
\StringTok{  }\KeywordTok{aes}\NormalTok{(}\DataTypeTok{x =}\NormalTok{ Cement)   }\OperatorTok{+}\StringTok{       }\CommentTok{# select variable to print... You can get really fancy here later}

\StringTok{  }\KeywordTok{geom_histogram}\NormalTok{()          }\CommentTok{# NEW: insert histogram}
\end{Highlighting}
\end{Shaded}

\begin{verbatim}
## `stat_bin()` using `bins = 30`. Pick better value with `binwidth`.
\end{verbatim}

\includegraphics{Concrete_Multivariate_Regression_Using_R_with_Tidyverse_files/figure-latex/unnamed-chunk-11-1.pdf}

(you may have gotten a warning about using the bin=X, you can adjust
it.)

Now quickly before moving on\ldots{} I am not keen on the grey
background with white lines.

There are a number of out-of-the-box
\href{https://ggplot2.tidyverse.org/reference/ggtheme.html}{``themes''}
for ggplot2.

I'm partial to theme\_bw() and theme\_light() but try the ones that you
prefer or stick with the default, theme\_gray().

These plots shown here are mine. You should fidget about so they are
\emph{yours} and so you can adapt to this new way of working with data.

\begin{Shaded}
\begin{Highlighting}[]
\KeywordTok{ggplot}\NormalTok{(}\DataTypeTok{data =}\NormalTok{ concrete) }\OperatorTok{+}\StringTok{ }\CommentTok{# invoke graphics environment using a given dataframe}
\StringTok{  }
\StringTok{  }\KeywordTok{theme_bw}\NormalTok{() }\OperatorTok{+}\StringTok{            }\CommentTok{# NEW: changing the plotting theme}
\StringTok{  }
\StringTok{  }\KeywordTok{aes}\NormalTok{(}\DataTypeTok{x =}\NormalTok{ Cement) }\OperatorTok{+}\StringTok{       }\CommentTok{# select variable to print... You can get really fancy here later}

\StringTok{  }\KeywordTok{geom_histogram}\NormalTok{()        }\CommentTok{# insert histogram (including controlling number of bins)}
\end{Highlighting}
\end{Shaded}

\begin{verbatim}
## `stat_bin()` using `bins = 30`. Pick better value with `binwidth`.
\end{verbatim}

\includegraphics{Concrete_Multivariate_Regression_Using_R_with_Tidyverse_files/figure-latex/unnamed-chunk-12-1.pdf}

My OCD hates axes where the labels don't envelop all of the data\ldots{}

We can fix that with
\href{https://ggplot2.tidyverse.org/reference/lims.html}{xlim() or
ylim()}

\begin{Shaded}
\begin{Highlighting}[]
\KeywordTok{ggplot}\NormalTok{(}\DataTypeTok{data =}\NormalTok{ concrete) }\OperatorTok{+}\StringTok{     }\CommentTok{# invoke graphics environment using a given dataframe}
\StringTok{  }
\StringTok{  }\KeywordTok{theme_bw}\NormalTok{() }\OperatorTok{+}\StringTok{                }\CommentTok{# changing the plotting theme}
\StringTok{  }
\StringTok{  }\KeywordTok{aes}\NormalTok{(}\DataTypeTok{x =}\NormalTok{ Cement) }\OperatorTok{+}\StringTok{           }\CommentTok{# select variable to print... You can get really fancy here later}
\StringTok{  }
\StringTok{  }\KeywordTok{xlim}\NormalTok{( }\DecValTok{100}\NormalTok{, }\DecValTok{400}\NormalTok{ ) }\OperatorTok{+}\StringTok{          }\CommentTok{# NEW: adding x-axis limits}

\StringTok{  }\KeywordTok{geom_histogram}\NormalTok{()            }\CommentTok{# insert histogram}
\end{Highlighting}
\end{Shaded}

\begin{verbatim}
## `stat_bin()` using `bins = 30`. Pick better value with `binwidth`.
\end{verbatim}

\begin{verbatim}
## Warning: Removed 1 rows containing missing values (geom_bar).
\end{verbatim}

\includegraphics{Concrete_Multivariate_Regression_Using_R_with_Tidyverse_files/figure-latex/unnamed-chunk-13-1.pdf}

How about changing the color of the fill in the bars\ldots{}

\href{https://www.nceas.ucsb.edu/~frazier/RSpatialGuides/colorPaletteCheatsheet.pdf}{You
really don't want to know about all the colors you can use.}

\begin{Shaded}
\begin{Highlighting}[]
\KeywordTok{ggplot}\NormalTok{(}\DataTypeTok{data =}\NormalTok{ concrete) }\OperatorTok{+}\StringTok{     }\CommentTok{# invoke graphics environment using a given dataframe}
\StringTok{  }
\StringTok{  }\KeywordTok{theme_bw}\NormalTok{() }\OperatorTok{+}\StringTok{                }\CommentTok{# changing the plotting theme}
\StringTok{  }
\StringTok{  }\KeywordTok{aes}\NormalTok{(}\DataTypeTok{x =}\NormalTok{ Cement) }\OperatorTok{+}\StringTok{           }\CommentTok{# select variable to print... You can get really fancy here later}
\StringTok{  }
\StringTok{  }\KeywordTok{xlim}\NormalTok{( }\DecValTok{100}\NormalTok{, }\DecValTok{400}\NormalTok{ ) }\OperatorTok{+}\StringTok{          }\CommentTok{# NEW: adding x-axis limits}

\StringTok{  }\KeywordTok{geom_histogram}\NormalTok{(}\DataTypeTok{fill=}\StringTok{"gray"}\NormalTok{) }\CommentTok{# EDIT: insert histogram (with a single chosen color)}
\end{Highlighting}
\end{Shaded}

\begin{verbatim}
## `stat_bin()` using `bins = 30`. Pick better value with `binwidth`.
\end{verbatim}

\begin{verbatim}
## Warning: Removed 1 rows containing missing values (geom_bar).
\end{verbatim}

\includegraphics{Concrete_Multivariate_Regression_Using_R_with_Tidyverse_files/figure-latex/unnamed-chunk-14-1.pdf}

Want to customize the labels and titles so we can have units?

You can add custom labels and titles!
(\url{https://www.nceas.ucsb.edu/~frazier/RSpatialGuides/colorPaletteCheatsheet.pdf})

For the superscripting in the x-axis label, I am using the
\href{http://vis.supstat.com/2013/04/mathematical-annotation-in-r/}{expression()}
tool in R.

\begin{Shaded}
\begin{Highlighting}[]
\KeywordTok{ggplot}\NormalTok{(}\DataTypeTok{data =}\NormalTok{ concrete) }\OperatorTok{+}\StringTok{     }\CommentTok{# invoke graphics environment using a given dataframe}
\StringTok{  }
\StringTok{  }\KeywordTok{theme_bw}\NormalTok{() }\OperatorTok{+}\StringTok{                }\CommentTok{# changing the plotting theme}
\StringTok{  }
\StringTok{  }\KeywordTok{aes}\NormalTok{(}\DataTypeTok{x =}\NormalTok{ Cement) }\OperatorTok{+}\StringTok{           }\CommentTok{# select variable to print... You can get really fancy here later}
\StringTok{  }
\StringTok{  }\KeywordTok{xlim}\NormalTok{( }\DecValTok{100}\NormalTok{, }\DecValTok{400}\NormalTok{ ) }\OperatorTok{+}\StringTok{          }\CommentTok{# adding x-axis limits}

\StringTok{  }\KeywordTok{ggtitle}\NormalTok{(}\StringTok{"Yeh Superplasticizer Tests"}\NormalTok{) }\OperatorTok{+}\StringTok{          }\CommentTok{# NEW : Custom Title}
\StringTok{  }
\StringTok{  }\KeywordTok{xlab}\NormalTok{(}\KeywordTok{expression}\NormalTok{(}\StringTok{'Cement Amount (kg m'}\OperatorTok{^-}\DecValTok{3}\OperatorTok{*}\StringTok{")"}\NormalTok{)) }\OperatorTok{+}\StringTok{ }\CommentTok{# NEW : Custom Axis Label}

\StringTok{  }\KeywordTok{geom_histogram}\NormalTok{(}\DataTypeTok{fill=}\StringTok{"gray"}\NormalTok{) }\CommentTok{# insert histogram (with a single chosen color)}
\end{Highlighting}
\end{Shaded}

\begin{verbatim}
## `stat_bin()` using `bins = 30`. Pick better value with `binwidth`.
\end{verbatim}

\begin{verbatim}
## Warning: Removed 1 rows containing missing values (geom_bar).
\end{verbatim}

\includegraphics{Concrete_Multivariate_Regression_Using_R_with_Tidyverse_files/figure-latex/unnamed-chunk-15-1.pdf}
And I could keep tweaking this graph all day, but good enough is good
enough so this is a good place to stop\ldots{}

We also can plot a few other fields with some trial and error..

\begin{Shaded}
\begin{Highlighting}[]
\CommentTok{# Histogram of Water}

\KeywordTok{ggplot}\NormalTok{(}\DataTypeTok{data =}\NormalTok{ concrete) }\OperatorTok{+}\StringTok{     }\CommentTok{# invoke graphics environment using a given dataframe}
\StringTok{  }
\StringTok{  }\KeywordTok{theme_bw}\NormalTok{() }\OperatorTok{+}\StringTok{                }\CommentTok{# changing the plotting theme}
\StringTok{  }
\StringTok{  }\KeywordTok{aes}\NormalTok{(}\DataTypeTok{x =}\NormalTok{ Water) }\OperatorTok{+}\StringTok{           }\CommentTok{# select variable to print... You can get really fancy here later}
\StringTok{  }
\StringTok{  }\KeywordTok{xlim}\NormalTok{( }\DecValTok{150}\NormalTok{, }\DecValTok{250}\NormalTok{ ) }\OperatorTok{+}\StringTok{          }\CommentTok{# adding x-axis limits}

\StringTok{  }\KeywordTok{ggtitle}\NormalTok{(}\StringTok{"Yeh Superplasticizer Tests"}\NormalTok{) }\OperatorTok{+}\StringTok{ }\CommentTok{#Custom Title}
\StringTok{  }
\StringTok{  }\KeywordTok{xlab}\NormalTok{(}\KeywordTok{expression}\NormalTok{(}\StringTok{'Water Amount (kg m'}\OperatorTok{^-}\DecValTok{3}\OperatorTok{*}\StringTok{")"}\NormalTok{)) }\OperatorTok{+}\StringTok{ }\CommentTok{# NEW : Custom Axis Label note use of superscripts from above}

\StringTok{  }\KeywordTok{geom_histogram}\NormalTok{(}\DataTypeTok{fill=}\StringTok{"blue"}\NormalTok{) }\CommentTok{# insert histogram (with a single chosen color)}
\end{Highlighting}
\end{Shaded}

\begin{verbatim}
## `stat_bin()` using `bins = 30`. Pick better value with `binwidth`.
\end{verbatim}

\includegraphics{Concrete_Multivariate_Regression_Using_R_with_Tidyverse_files/figure-latex/unnamed-chunk-16-1.pdf}

\begin{Shaded}
\begin{Highlighting}[]
\CommentTok{# Histogram of Strength}

\KeywordTok{ggplot}\NormalTok{(}\DataTypeTok{data =}\NormalTok{ concrete) }\OperatorTok{+}\StringTok{     }\CommentTok{# invoke graphics environment using a given dataframe}
\StringTok{  }
\StringTok{  }\KeywordTok{theme_bw}\NormalTok{() }\OperatorTok{+}\StringTok{                }\CommentTok{# changing the plotting theme}
\StringTok{  }
\StringTok{  }\KeywordTok{aes}\NormalTok{(}\DataTypeTok{x =}\NormalTok{ Compressive_Strength_28dy) }\OperatorTok{+}\StringTok{ }\CommentTok{# select variable to print... You can get really fancy here later}
\StringTok{  }
\StringTok{  }\KeywordTok{xlim}\NormalTok{( }\DecValTok{10}\NormalTok{, }\DecValTok{60}\NormalTok{ ) }\OperatorTok{+}\StringTok{          }\CommentTok{# adding x-axis limits}

\StringTok{  }\KeywordTok{ggtitle}\NormalTok{(}\StringTok{"Yeh Superplasticizer Tests"}\NormalTok{) }\OperatorTok{+}\StringTok{ }\CommentTok{#Custom Title}
\StringTok{  }
\StringTok{  }\KeywordTok{xlab}\NormalTok{(}\StringTok{"28-dy Compressive Strength (MPa)"}\NormalTok{) }\OperatorTok{+}\StringTok{ }\CommentTok{# NEW : Custom Axis Label}

\StringTok{  }\KeywordTok{geom_histogram}\NormalTok{(}\DataTypeTok{fill=}\StringTok{"red"}\NormalTok{) }\CommentTok{# insert histogram (with a single chosen color)}
\end{Highlighting}
\end{Shaded}

\begin{verbatim}
## `stat_bin()` using `bins = 30`. Pick better value with `binwidth`.
\end{verbatim}

\begin{verbatim}
## Warning: Removed 1 rows containing missing values (geom_bar).
\end{verbatim}

\includegraphics{Concrete_Multivariate_Regression_Using_R_with_Tidyverse_files/figure-latex/unnamed-chunk-17-1.pdf}
\#\# 5.2 Distribution Plot {[}not so good an{]} Example

There are some other plots that we can use to describe our data.

Here to play with them we will take a quick step back and address that
``tidy'''ed (should that say ``tidied''?) dataframe ``concrete\_tidy''

We can now use all the parameters in the ``tidy'' (long) data frame to
print by specific traits.

\begin{Shaded}
\begin{Highlighting}[]
\KeywordTok{ggplot}\NormalTok{(}\DataTypeTok{data =}\NormalTok{ concrete_tidy) }\OperatorTok{+}\StringTok{            }\CommentTok{# invoke graphics environment using a given dataframe}
\StringTok{  }
\StringTok{  }\KeywordTok{theme_bw}\NormalTok{() }\OperatorTok{+}\StringTok{                            }\CommentTok{# changing the plotting theme}
\StringTok{  }
\StringTok{  }\KeywordTok{aes}\NormalTok{(}\DataTypeTok{x      =}\NormalTok{ Value,                     }\CommentTok{# map x-axis value}
      \DataTypeTok{color  =}\NormalTok{ Parameter) }\OperatorTok{+}\StringTok{               }\CommentTok{# map colors for different quality}
\StringTok{  }
\StringTok{  }\KeywordTok{ggtitle}\NormalTok{(}\StringTok{"Yeh Superplasticizer Tests"}\NormalTok{) }\OperatorTok{+}\StringTok{ }\CommentTok{# Custom Title}
\StringTok{  }
\StringTok{  }\KeywordTok{xlab}\NormalTok{(}\StringTok{"Value"}\NormalTok{) }\OperatorTok{+}\StringTok{                         }\CommentTok{#  Custom Axis Label}

\StringTok{  }\KeywordTok{geom_density}\NormalTok{()                          }\CommentTok{# insert crete a relative density plot }
\end{Highlighting}
\end{Shaded}

\includegraphics{Concrete_Multivariate_Regression_Using_R_with_Tidyverse_files/figure-latex/unnamed-chunk-18-1.pdf}

In the past, I've gotten good results with this but in this case, I
think it's too messy in part due to the disparity in the dynamic range
of our parameters.

\hypertarget{box-whisker-plot-example}{%
\subsection{5.3. Box-Whisker Plot
Example}\label{box-whisker-plot-example}}

How about leveraging a box whisker? (I'm using only the independent
variables this time.)

\begin{Shaded}
\begin{Highlighting}[]
\KeywordTok{ggplot}\NormalTok{(}\DataTypeTok{data =}\NormalTok{ concrete_independent) }\OperatorTok{+}\StringTok{      }\CommentTok{# EDIT Changing dataframe}
\StringTok{  }
\StringTok{  }\KeywordTok{theme_bw}\NormalTok{( ) }\OperatorTok{+}\StringTok{                            }\CommentTok{# changing the plotting theme}
\StringTok{  }
\StringTok{  }\KeywordTok{theme}\NormalTok{(}\DataTypeTok{axis.text.x =} \KeywordTok{element_blank}\NormalTok{()) }\OperatorTok{+}\StringTok{   }\CommentTok{# adding an extra trait to the x-axis}
\StringTok{                                           }\CommentTok{# to not print labels on the x-axis }
\StringTok{                                           }\CommentTok{# (the labels overlap and doesn't look}
\StringTok{                                           }\CommentTok{# pretty...)}
\StringTok{  }
\StringTok{  }\KeywordTok{aes}\NormalTok{(}\DataTypeTok{y      =}\NormalTok{ Value,                     }\CommentTok{# map y-axis value}
      \DataTypeTok{x      =}\NormalTok{ Parameter,                 }\CommentTok{# map x-axis value}
      \DataTypeTok{color  =}\NormalTok{ Parameter) }\OperatorTok{+}\StringTok{               }\CommentTok{# map colors for different quality}
\StringTok{  }
\StringTok{  }\KeywordTok{ggtitle}\NormalTok{(}\DataTypeTok{label    =} \StringTok{"Yeh Superplasticizer Tests"}\NormalTok{,}
          \DataTypeTok{subtitle =} \StringTok{"Concrete Test Components"}\NormalTok{) }\OperatorTok{+}\StringTok{ }\CommentTok{# Custom Title}
\StringTok{  }
\StringTok{  }\KeywordTok{ylab}\NormalTok{(}\KeywordTok{expression}\NormalTok{(}\StringTok{'Amount (kg m'}\OperatorTok{^-}\DecValTok{3}\OperatorTok{*}\StringTok{")"}\NormalTok{)) }\OperatorTok{+}\StringTok{ }\CommentTok{# EDIT : Changing Custom Axis Label}

\StringTok{  }\KeywordTok{geom_boxplot}\NormalTok{()                          }\CommentTok{# insert crete a relative density plot }
\end{Highlighting}
\end{Shaded}

\includegraphics{Concrete_Multivariate_Regression_Using_R_with_Tidyverse_files/figure-latex/unnamed-chunk-19-1.pdf}

What about our dependant variables? We can start by changing the data
frame\ldots{}

\begin{Shaded}
\begin{Highlighting}[]
\KeywordTok{ggplot}\NormalTok{(}\DataTypeTok{data =}\NormalTok{ concrete_dependent) }\OperatorTok{+}\StringTok{      }\CommentTok{# EDIT Changing dataframe}
\StringTok{  }
\StringTok{  }\KeywordTok{theme_bw}\NormalTok{( ) }\OperatorTok{+}\StringTok{                            }\CommentTok{# changing the plotting theme}
\StringTok{  }
\StringTok{  }\KeywordTok{theme}\NormalTok{(}\DataTypeTok{axis.text.x =} \KeywordTok{element_blank}\NormalTok{()) }\OperatorTok{+}\StringTok{   }\CommentTok{# adding an extra trait to the x-axis}
\StringTok{                                           }\CommentTok{# to not print labels on the x-axis }
\StringTok{                                           }\CommentTok{# (the labels overlap and doesn't look}
\StringTok{                                           }\CommentTok{# pretty...)}
\StringTok{  }
\StringTok{  }\KeywordTok{aes}\NormalTok{(}\DataTypeTok{y      =}\NormalTok{ Value,                     }\CommentTok{# map y-axis value}
      \DataTypeTok{x      =}\NormalTok{ Parameter,                 }\CommentTok{# map x-axis value}
      \DataTypeTok{color  =}\NormalTok{ Parameter) }\OperatorTok{+}\StringTok{               }\CommentTok{# map colors for different quality}
\StringTok{  }
\StringTok{  }\KeywordTok{ggtitle}\NormalTok{(}\DataTypeTok{label    =} \StringTok{"Yeh Superplasticizer Tests"}\NormalTok{,}
          \DataTypeTok{subtitle =} \StringTok{"Concrete Test Results"}\NormalTok{) }\OperatorTok{+}\StringTok{ }\CommentTok{# Custom Title}
\StringTok{  }
\StringTok{  }\KeywordTok{ylab}\NormalTok{(}\StringTok{"Values"}\NormalTok{) }\OperatorTok{+}

\StringTok{  }\KeywordTok{geom_boxplot}\NormalTok{()                          }\CommentTok{# insert crete a relative density plot }
\end{Highlighting}
\end{Shaded}

\includegraphics{Concrete_Multivariate_Regression_Using_R_with_Tidyverse_files/figure-latex/unnamed-chunk-20-1.pdf}

Want units? That's a little tougher here since the units differ by
parameter. We can force the values to into new names though.

\begin{Shaded}
\begin{Highlighting}[]
\KeywordTok{ggplot}\NormalTok{(}\DataTypeTok{data =}\NormalTok{ concrete_dependent) }\OperatorTok{+}\StringTok{      }\CommentTok{# EDIT Changing dataframe}
\StringTok{  }
\StringTok{  }\KeywordTok{theme_bw}\NormalTok{( ) }\OperatorTok{+}\StringTok{                            }\CommentTok{# changing the plotting theme}
\StringTok{  }
\StringTok{  }\KeywordTok{theme}\NormalTok{(}\DataTypeTok{axis.text.x =} \KeywordTok{element_blank}\NormalTok{()) }\OperatorTok{+}\StringTok{   }\CommentTok{# adding an extra trait to the x-axis}
\StringTok{                                           }\CommentTok{# to not print labels on the x-axis }
\StringTok{                                           }\CommentTok{# (the labels overlap and doesn't look}
\StringTok{                                           }\CommentTok{# pretty...)}
\StringTok{  }
\StringTok{  }\KeywordTok{aes}\NormalTok{(}\DataTypeTok{y      =}\NormalTok{ Value,                     }\CommentTok{# map y-axis value}
      \DataTypeTok{x      =}\NormalTok{ Parameter,                 }\CommentTok{# map x-axis value}
      \DataTypeTok{color  =}\NormalTok{ Parameter) }\OperatorTok{+}\StringTok{               }\CommentTok{# map colors for different quality}
\StringTok{  }
\StringTok{  }\KeywordTok{ggtitle}\NormalTok{(}\DataTypeTok{label    =} \StringTok{"Yeh Superplasticizer Tests"}\NormalTok{,}
          \DataTypeTok{subtitle =} \StringTok{"Concrete Test Results"}\NormalTok{) }\OperatorTok{+}\StringTok{ }\CommentTok{# Custom Title}
\StringTok{  }
\StringTok{  }\KeywordTok{ylab}\NormalTok{(}\StringTok{"Values"}\NormalTok{) }\OperatorTok{+}

\StringTok{  }\CommentTok{# NEW: It says scale color but "color" is how we are distinguishing}
\StringTok{  }\CommentTok{#      out boxplots (as seen in the mapping/aes command)}
\StringTok{  }\CommentTok{#      we can then use the same plot order above to rewrite the labels}
\StringTok{  }\CommentTok{#      (likewise we could change the plot order and of coruse the colors.)}
\StringTok{  }\KeywordTok{scale_color_discrete}\NormalTok{(}\DataTypeTok{labels =} \KeywordTok{c}\NormalTok{(}\StringTok{"Slump (cm)"}\NormalTok{,}
                                  \StringTok{"Flow (cm)"}\NormalTok{, }
                                  \StringTok{"28dy-Compresional Stress (mPa)"}\NormalTok{)) }\OperatorTok{+}\StringTok{ }
\StringTok{  }
\StringTok{  }\KeywordTok{geom_boxplot}\NormalTok{() }\CommentTok{# insert crete a relative density plot }
\end{Highlighting}
\end{Shaded}

\includegraphics{Concrete_Multivariate_Regression_Using_R_with_Tidyverse_files/figure-latex/unnamed-chunk-21-1.pdf}

\hypertarget{violin-plot-example}{%
\subsection{5.4. Violin Plot Example}\label{violin-plot-example}}

How about leveraging a ``violin'' plot? A violin plot's width swells in
areas with more observations and contracts with sparser data so it is
like looking at a probability distribution.

\begin{Shaded}
\begin{Highlighting}[]
\KeywordTok{ggplot}\NormalTok{(}\DataTypeTok{data =}\NormalTok{ concrete_independent) }\OperatorTok{+}\StringTok{      }\CommentTok{# EDIT Changing dataframe}
\StringTok{  }
\StringTok{  }\KeywordTok{theme_bw}\NormalTok{( ) }\OperatorTok{+}\StringTok{                            }\CommentTok{# changing the plotting theme}
\StringTok{  }
\StringTok{  }\KeywordTok{theme}\NormalTok{(}\DataTypeTok{axis.text.x =} \KeywordTok{element_blank}\NormalTok{()) }\OperatorTok{+}\StringTok{   }\CommentTok{# adding an extra trait to the x-axis}
\StringTok{                                           }\CommentTok{# to not print labels on the x-axis }
\StringTok{                                           }\CommentTok{# (the labels overlap and doesn't look}
\StringTok{                                           }\CommentTok{# pretty...)}
\StringTok{  }
\StringTok{  }\KeywordTok{aes}\NormalTok{(}\DataTypeTok{y      =}\NormalTok{ Value,                     }\CommentTok{# map y-axis value}
      \DataTypeTok{x      =}\NormalTok{ Parameter,                 }\CommentTok{# map x-axis value}
      \DataTypeTok{color  =}\NormalTok{ Parameter) }\OperatorTok{+}\StringTok{               }\CommentTok{# map colors for different quality}
\StringTok{  }
\StringTok{  }\KeywordTok{ggtitle}\NormalTok{(}\DataTypeTok{label    =} \StringTok{"Yeh Superplasticizer Tests"}\NormalTok{,}
          \DataTypeTok{subtitle =} \StringTok{"Concrete Test Components"}\NormalTok{) }\OperatorTok{+}\StringTok{ }\CommentTok{# Custom Title}
\StringTok{  }
\StringTok{  }\KeywordTok{ylab}\NormalTok{(}\KeywordTok{expression}\NormalTok{(}\StringTok{'Amount (kg m'}\OperatorTok{^-}\DecValTok{3}\OperatorTok{*}\StringTok{")"}\NormalTok{)) }\OperatorTok{+}\StringTok{ }\CommentTok{#  Changing Custom Axis Label}

\StringTok{  }\KeywordTok{geom_violin}\NormalTok{(}\DataTypeTok{scale=}\StringTok{"width"}\NormalTok{) }\CommentTok{# EDIT: change to a violin plot }
\end{Highlighting}
\end{Shaded}

\includegraphics{Concrete_Multivariate_Regression_Using_R_with_Tidyverse_files/figure-latex/unnamed-chunk-22-1.pdf}

\begin{Shaded}
\begin{Highlighting}[]
                             \CommentTok{#   the width argument }
                             \CommentTok{# gives every plot the same width}
\end{Highlighting}
\end{Shaded}

and\ldots{}

\begin{Shaded}
\begin{Highlighting}[]
\KeywordTok{ggplot}\NormalTok{(}\DataTypeTok{data =}\NormalTok{ concrete_dependent) }\OperatorTok{+}\StringTok{      }\CommentTok{# EDIT Changing dataframe}
\StringTok{  }
\StringTok{  }\KeywordTok{theme_bw}\NormalTok{( ) }\OperatorTok{+}\StringTok{                            }\CommentTok{# changing the plotting theme}
\StringTok{  }
\StringTok{  }\KeywordTok{theme}\NormalTok{(}\DataTypeTok{axis.text.x =} \KeywordTok{element_blank}\NormalTok{()) }\OperatorTok{+}\StringTok{   }\CommentTok{# adding an extra trait to the x-axis}
\StringTok{                                           }\CommentTok{# to not print labels on the x-axis }
\StringTok{                                           }\CommentTok{# (the labels overlap and doesn't look}
\StringTok{                                           }\CommentTok{# pretty...)}
\StringTok{  }
\StringTok{  }\KeywordTok{aes}\NormalTok{(}\DataTypeTok{y      =}\NormalTok{ Value,                     }\CommentTok{# map y-axis value}
      \DataTypeTok{x      =}\NormalTok{ Parameter,                 }\CommentTok{# map x-axis value}
      \DataTypeTok{color  =}\NormalTok{ Parameter) }\OperatorTok{+}\StringTok{               }\CommentTok{# map colors for different quality}
\StringTok{  }
\StringTok{  }\KeywordTok{ggtitle}\NormalTok{(}\DataTypeTok{label    =} \StringTok{"Yeh Superplasticizer Tests"}\NormalTok{,}
          \DataTypeTok{subtitle =} \StringTok{"Concrete Test Results"}\NormalTok{) }\OperatorTok{+}\StringTok{ }\CommentTok{# Custom Title}
\StringTok{  }
\StringTok{  }\KeywordTok{ylab}\NormalTok{(}\StringTok{"Values"}\NormalTok{) }\OperatorTok{+}

\StringTok{  }\CommentTok{# NEW: It says scale color but "color" is how we are distinguishing}
\StringTok{  }\CommentTok{#      out boxplots (as seen in the mapping/aes command)}
\StringTok{  }\CommentTok{#      we can then use the same plot order above to rewrite the labels}
\StringTok{  }\CommentTok{#      (likewise we could change the plot order and of coruse the colors.)}
\StringTok{  }\KeywordTok{scale_color_discrete}\NormalTok{(}\DataTypeTok{labels =} \KeywordTok{c}\NormalTok{(}\StringTok{"Slump (cm)"}\NormalTok{,}
                                  \StringTok{"Flow (cm)"}\NormalTok{, }
                                  \StringTok{"28dy-Compresional Stress (mPa)"}\NormalTok{)) }\OperatorTok{+}\StringTok{ }
\StringTok{  }

\StringTok{  }\KeywordTok{geom_violin}\NormalTok{(}\DataTypeTok{scale=}\StringTok{"width"}\NormalTok{) }\CommentTok{# EDIT: change to a violin plot }
\end{Highlighting}
\end{Shaded}

\includegraphics{Concrete_Multivariate_Regression_Using_R_with_Tidyverse_files/figure-latex/unnamed-chunk-23-1.pdf}

\begin{Shaded}
\begin{Highlighting}[]
                             \CommentTok{#   the width argument }
                             \CommentTok{# gives every plot the same width  }
\end{Highlighting}
\end{Shaded}

This is basically the above ``density'' plot but ``looking down'' as
with a box plot. Also here we are trimming the plot so that when we
leave the range of any of the data points, the ``violins'' are
truncated.

\hypertarget{stacked-column-or-bar-plot-example}{%
\subsection{5.5. Stacked Column or Bar Plot
Example}\label{stacked-column-or-bar-plot-example}}

We also can do bar plots or stacked column plots. The one produced here
shows the combined components by test unit.

\begin{Shaded}
\begin{Highlighting}[]
\KeywordTok{ggplot}\NormalTok{(}\DataTypeTok{data =}\NormalTok{ concrete_independent) }\OperatorTok{+}\StringTok{      }\CommentTok{# EDIT Changing dataframe}
\StringTok{  }
\StringTok{  }\KeywordTok{theme_bw}\NormalTok{( ) }\OperatorTok{+}\StringTok{                            }\CommentTok{# changing the plotting theme}

\StringTok{  }
\StringTok{  }\KeywordTok{aes}\NormalTok{(}\DataTypeTok{x     =}\NormalTok{ Test_Number,}
      \DataTypeTok{y     =}\NormalTok{ Value,}
      \DataTypeTok{fill  =}\NormalTok{ Parameter) }\OperatorTok{+}\StringTok{               }\CommentTok{# map colors for different quality}
\StringTok{  }
\StringTok{  }\KeywordTok{ggtitle}\NormalTok{(}\DataTypeTok{label    =} \StringTok{"Yeh Superplasticizer Tests"}\NormalTok{,}
          \DataTypeTok{subtitle =} \StringTok{"Concrete Test Components"}\NormalTok{) }\OperatorTok{+}\StringTok{ }\CommentTok{# Custom Title}
\StringTok{  }
\StringTok{  }\KeywordTok{ylab}\NormalTok{(}\KeywordTok{expression}\NormalTok{(}\StringTok{'Amount (kg m'}\OperatorTok{^-}\DecValTok{3}\OperatorTok{*}\StringTok{")"}\NormalTok{)) }\OperatorTok{+}\StringTok{ }\CommentTok{#  Changing Custom Axis Label}

\StringTok{  }\KeywordTok{geom_col}\NormalTok{(}\DataTypeTok{position =} \StringTok{"stack"}\NormalTok{,  }\CommentTok{# new, create a stacekd column graph }
           \DataTypeTok{width    =} \FloatTok{1.0}\NormalTok{    )  }\CommentTok{# with no space between columns}
\end{Highlighting}
\end{Shaded}

\includegraphics{Concrete_Multivariate_Regression_Using_R_with_Tidyverse_files/figure-latex/unnamed-chunk-24-1.pdf}

\hypertarget{correlation-of-variables}{%
\section{6. Correlation of Variables}\label{correlation-of-variables}}

\hypertarget{correlating-and-then-fitting-cement-to-compressive-strength}{%
\subsection{6.1. Correlating and then Fitting Cement to Compressive
Strength}\label{correlating-and-then-fitting-cement-to-compressive-strength}}

Let's start by doing a ``simple''" plot . In this case since I already
know the answer because the spreadsheet also has a table of how well our
independent variables correlate against the dependent variables (e.g.,
Slump, Flow, or in our case Strength). The Cement correlates the best
against Compressive Strength (OK, truth be told, it correlates the least
badly).

We can actually do this with a correlate function,
\href{https://www.rdocumentation.org/packages/stats/versions/3.4.3/topics/cor}{cor()}\ldots{}

To grab a value in the table ``concrete'' we call the data frame
(concrete) and the variable name (Cement or Water vs
Compressive\_Strength\_28dy), separating the frame and variable names by
a \$ sign.

\begin{Shaded}
\begin{Highlighting}[]
\KeywordTok{print}\NormalTok{(}\StringTok{"Cement vs Compressive Strength Correlation, r"}\NormalTok{)}
\end{Highlighting}
\end{Shaded}

\begin{verbatim}
## [1] "Cement vs Compressive Strength Correlation, r"
\end{verbatim}

\begin{Shaded}
\begin{Highlighting}[]
\KeywordTok{cor}\NormalTok{(}\DataTypeTok{x =}\NormalTok{ concrete}\OperatorTok{$}\NormalTok{Cement,                    }\CommentTok{# the x-value }
    \DataTypeTok{y =}\NormalTok{ concrete}\OperatorTok{$}\NormalTok{Compressive_Strength_28dy, }\CommentTok{# the y-value}
    \DataTypeTok{method =} \StringTok{"pearson"}                      \CommentTok{# method of correlation}
\NormalTok{    )}
\end{Highlighting}
\end{Shaded}

\begin{verbatim}
## [1] 0.45
\end{verbatim}

or if you like to do everything at once\ldots{}

\begin{Shaded}
\begin{Highlighting}[]
\CommentTok{# calculate all correlation values against each other}

\NormalTok{correlation_matrix =}\StringTok{ }\KeywordTok{cor}\NormalTok{(}\DataTypeTok{x      =}\NormalTok{ concrete, }\CommentTok{# using our dataframe to correlate evything}
                         \DataTypeTok{method =} \StringTok{"pearson"}\NormalTok{ )}

\KeywordTok{tbl_df}\NormalTok{(correlation_matrix)}
\end{Highlighting}
\end{Shaded}

\begin{verbatim}
## # A tibble: 11 x 11
##    Test_Number  Cement    Slag Fly_Ash   Water Superplasticizer
##          <dbl>   <dbl>   <dbl>   <dbl>   <dbl>            <dbl>
##  1     1       -0.0316 -0.0798  0.341  -0.138           -0.335 
##  2    -0.0316   1      -0.244  -0.487   0.221           -0.106 
##  3    -0.0798  -0.244   1      -0.323  -0.0268           0.307 
##  4     0.341   -0.487  -0.323   1      -0.241           -0.144 
##  5    -0.138    0.221  -0.0268 -0.241   1               -0.155 
##  6    -0.335   -0.106   0.307  -0.144  -0.155            1     
##  7     0.222   -0.310  -0.224   0.173  -0.602           -0.104 
##  8    -0.314    0.0570 -0.184  -0.283   0.115            0.0583
##  9     0.0374   0.146  -0.284  -0.119   0.467           -0.213 
## 10     0.00866  0.186  -0.327  -0.0554  0.632           -0.176 
## 11     0.186    0.446  -0.332   0.444  -0.254           -0.0379
## # ... with 5 more variables: Coarse_Aggregates <dbl>,
## #   Fine_Aggregates <dbl>, Slump <dbl>, Flow <dbl>,
## #   Compressive_Strength_28dy <dbl>
\end{verbatim}

Lots of numbers\ldots{} not all that insightful on their own\ldots{}

You also can graph the look-n-feel of what all of the different
correlations are\ldots{} (it works best with a much smaller number of
variables)

\begin{Shaded}
\begin{Highlighting}[]
  \CommentTok{# draw a coorelation graphic...}

  \KeywordTok{corrplot}\NormalTok{(}\DataTypeTok{corr   =}\NormalTok{ correlation_matrix,}
           \DataTypeTok{type   =} \StringTok{"upper"}\NormalTok{)}
\end{Highlighting}
\end{Shaded}

\includegraphics{Concrete_Multivariate_Regression_Using_R_with_Tidyverse_files/figure-latex/unnamed-chunk-27-1.pdf}
We can now see for example that cement, slag, and fly ash amounts have a
nominal but not thrilling correlation to compression strength while
water has a good correlation with the resulting slump values. One thing
that this does \emph{not} show is how well these parameters play with
other parameters. As we'll see when all of our independent values are
working together we'll discover that cement and water, followed by fly
ash and coarse aggregates will, together, contribute the most of our
independent parameters in calculating the compressive strength.

\hypertarget{scatter-plot-example}{%
\subsection{6.2. Scatter Plot Example}\label{scatter-plot-example}}

But for now, let's plot plot the Cement amount against Compressive
Strength

\begin{Shaded}
\begin{Highlighting}[]
\CommentTok{# Making a simple X-Y scatterplot.}

\KeywordTok{ggplot}\NormalTok{(}\DataTypeTok{data =}\NormalTok{ concrete) }\OperatorTok{+}\StringTok{                }\CommentTok{# invoke graphics environment using a given dataframe}
\StringTok{  }
\StringTok{  }\KeywordTok{theme_bw}\NormalTok{( ) }\OperatorTok{+}\StringTok{                           }\CommentTok{# changing the plotting theme}
\StringTok{  }
\StringTok{  }\KeywordTok{aes}\NormalTok{(}\DataTypeTok{x      =}\NormalTok{ Cement,                       }\CommentTok{# x-value}
      \DataTypeTok{y      =}\NormalTok{ Compressive_Strength_28dy) }\OperatorTok{+}\StringTok{  }\CommentTok{# y-value}

\StringTok{  }\KeywordTok{ggtitle}\NormalTok{(}\StringTok{"Yeh Superplasticizer Tests"}\NormalTok{) }\OperatorTok{+}\StringTok{    }\CommentTok{# Custom Title}
\StringTok{  }
\StringTok{  }\KeywordTok{xlab}\NormalTok{(}\KeywordTok{expression}\NormalTok{(}\StringTok{'Cement Amount (kg m'}\OperatorTok{^}\DecValTok{3}\OperatorTok{*}\StringTok{")"}\NormalTok{)) }\OperatorTok{+}\StringTok{   }\CommentTok{# x-label}
\StringTok{  }\KeywordTok{ylab}\NormalTok{(}\StringTok{"28-dy Compressive Strength (MPa)"}\NormalTok{)      }\OperatorTok{+}\StringTok{   }\CommentTok{# y-label}

\StringTok{  }\KeywordTok{geom_point}\NormalTok{(}\DataTypeTok{colour=}\StringTok{"grey"}\NormalTok{)   }\CommentTok{# EDIT: plot points the color keyword part was}
\end{Highlighting}
\end{Shaded}

\includegraphics{Concrete_Multivariate_Regression_Using_R_with_Tidyverse_files/figure-latex/unnamed-chunk-28-1.pdf}

\begin{Shaded}
\begin{Highlighting}[]
                              \CommentTok{#       writen by an anglophile!}
\end{Highlighting}
\end{Shaded}

Here's a cute trick: Could we color those dots by a variable?

Sure!

\begin{Shaded}
\begin{Highlighting}[]
\CommentTok{# Making a simple X-Y scatterplot now coloured by another parameter}

\KeywordTok{ggplot}\NormalTok{(}\DataTypeTok{data =}\NormalTok{ concrete) }\OperatorTok{+}\StringTok{                }\CommentTok{# invoke graphics environment using a given dataframe}
\StringTok{  }
\StringTok{  }\KeywordTok{theme_bw}\NormalTok{( ) }\OperatorTok{+}\StringTok{                           }\CommentTok{# changing the plotting theme}
\StringTok{  }
\StringTok{  }\KeywordTok{aes}\NormalTok{(}\DataTypeTok{x      =}\NormalTok{ Cement,                       }\CommentTok{# x-value}
      \DataTypeTok{y      =}\NormalTok{ Compressive_Strength_28dy,    }\CommentTok{# y-value}
      \DataTypeTok{color  =}\NormalTok{ Superplasticizer)          }\OperatorTok{+}\StringTok{  }\CommentTok{# ADD: we can color by a variable!}

\StringTok{  }\KeywordTok{ggtitle}\NormalTok{(}\StringTok{"Yeh Superplasticizer Tests"}\NormalTok{) }\OperatorTok{+}\StringTok{    }\CommentTok{# Custom Title}
\StringTok{  }
\StringTok{  }\KeywordTok{xlab}\NormalTok{(}\KeywordTok{expression}\NormalTok{(}\StringTok{'Cement Amount (kg m'}\OperatorTok{^}\DecValTok{3}\OperatorTok{*}\StringTok{")"}\NormalTok{)) }\OperatorTok{+}\StringTok{   }\CommentTok{# x-label}
\StringTok{  }\KeywordTok{ylab}\NormalTok{(}\StringTok{"28-dy Compressive Strength (MPa)"}\NormalTok{)      }\OperatorTok{+}\StringTok{   }\CommentTok{# y-label}

\StringTok{  }\KeywordTok{geom_point}\NormalTok{() }\OperatorTok{+}\StringTok{  }\CommentTok{# plot points }
\StringTok{  }\KeywordTok{scale_color_distiller}\NormalTok{(}\DataTypeTok{palette =} \StringTok{"Spectral"}\NormalTok{) }\CommentTok{# NEW: pick a custom "colour" palate.}
\end{Highlighting}
\end{Shaded}

\includegraphics{Concrete_Multivariate_Regression_Using_R_with_Tidyverse_files/figure-latex/unnamed-chunk-29-1.pdf}

Love overkill without any distinct numerical score and look at how
everything in your data set correlates with every other
variables\ldots{}?

Try
\href{https://www.rdocumentation.org/packages/graphics/versions/3.5.1/topics/pairs}{pairs()}

(I like the corrplot function better!)

\begin{Shaded}
\begin{Highlighting}[]
\CommentTok{# way too many tiny plots!}

\KeywordTok{pairs}\NormalTok{(}\DataTypeTok{x   =}\NormalTok{ concrete, }\CommentTok{# do everything in the dataframe}
      \DataTypeTok{pch =} \StringTok{"."}\NormalTok{)      }\CommentTok{# plot dots (the default is circles)}
\end{Highlighting}
\end{Shaded}

\includegraphics{Concrete_Multivariate_Regression_Using_R_with_Tidyverse_files/figure-latex/unnamed-chunk-30-1.pdf}

(Obviously the more variables in your dataframe the messier it gets!)

\hypertarget{creating-our-linear-model-and-calibrating-it}{%
\subsection{6.3. Creating our linear model and ``calibrating''
it}\label{creating-our-linear-model-and-calibrating-it}}

We weren't all that thrilled with the correlation between these
components and strength but let's go ahead and demonstrate a regression.

But let's move on and create a regression model from this.

Here we will use the
\href{https://www.rdocumentation.org/packages/stats/versions/3.4.3/topics/lm}{lm()}
(linear model) function from the MASS package.

For the regression formula

\(\widehat{y}(x) = {\alpha_0}+{\alpha_1}\ x\)

or

\(\widehat{Strength}(concrete) = {\alpha_0}+{\alpha_1}\ concrete\)

the ``prototype'' (formula) for the function is written as \ldots{}

``Y \textasciitilde{} X'' (with the y-intercept implicit in the
formula\ldots{} you don't put it in but it'll be there when you're
done.)

The above syntax is works like this\ldots{}.

Dependent Variable {[}\textasciitilde{} is a function of {]} Independent
Variable {[}and any other parameter you need gets added with a plus{]}

If this were a \(\widehat{y}(x)={\alpha_0}+{\alpha_0}\ x^3\), then the
prototype for the function would be y \textasciitilde{} x\^{}3

This will hopefully make more sense as we continue!

\emph{(lm and similar linear regression functions don't play well with
units.)}

\begin{Shaded}
\begin{Highlighting}[]
\NormalTok{linear_model.S_v_c =}\StringTok{  }\KeywordTok{lm}\NormalTok{(}\DataTypeTok{formula =}\NormalTok{ Compressive_Strength_28dy }\OperatorTok{~}\StringTok{ }\NormalTok{Cement, }\CommentTok{# your formula y ~ x}
                         \DataTypeTok{data    =}\NormalTok{ concrete)                           }\CommentTok{# the data frame}
\end{Highlighting}
\end{Shaded}

Let's see what we have\ldots{} This summary command will provide the
details of the lm() function's important results

For us we want to see the Y-Intercept {[}the (Intercept) under
``Estimate''{]} and the slope that goes with our independent value
(``Concrete'' under ``Estimate'')

The Standard Error of the Estimate is there (Residual Standard Error) as
is the Coefficient of Determination (Multiple R-squared)

We'll talk about a few of the other features when we do the larger
multivariate regression

\begin{Shaded}
\begin{Highlighting}[]
 \KeywordTok{summary}\NormalTok{(}\DataTypeTok{object =}\NormalTok{ linear_model.S_v_c)}
\end{Highlighting}
\end{Shaded}

\begin{verbatim}
## 
## Call:
## lm(formula = Compressive_Strength_28dy ~ Cement, data = concrete)
## 
## Residuals:
##     Min      1Q  Median      3Q     Max 
## -15.134  -5.313   0.832   5.155  17.968 
## 
## Coefficients:
##             Estimate Std. Error t value Pr(>|t|)    
## (Intercept) 25.85676    2.15022      12  < 2e-16 ***
## Cement       0.04429    0.00885       5  2.4e-06 ***
## ---
## Signif. codes:  0 '***' 0.001 '**' 0.01 '*' 0.05 '.' 0.1 ' ' 1
## 
## Residual standard error: 7 on 101 degrees of freedom
## Multiple R-squared:  0.199,  Adjusted R-squared:  0.191 
## F-statistic:   25 on 1 and 101 DF,  p-value: 2.38e-06
\end{verbatim}

In the above output, the asterisk identify the most significant
independent variables. Here it's trivial even though this is a terrible
relationship between cement and strength. Later we will use all of our
available independent variables and the use of these asterisks will
become more important.

Want to plot it?

Good news?

Like Excel, you have some automated features to give you quick
satisfaction and happiness. More still, it will give you confidence
limits.

For this we use an extension to the graphics package called
\href{https://ggplot2.tidyverse.org/reference/geom_smooth.html}{geom\_smooth()}

\begin{Shaded}
\begin{Highlighting}[]
\CommentTok{# Making a simple X-Y scatterplot and adding a regression to it}

\KeywordTok{ggplot}\NormalTok{(}\DataTypeTok{data =}\NormalTok{ concrete) }\OperatorTok{+}\StringTok{                }\CommentTok{# invoke graphics environment using a given dataframe}
\StringTok{  }
\StringTok{  }\KeywordTok{theme_bw}\NormalTok{( ) }\OperatorTok{+}\StringTok{                           }\CommentTok{# changing the plotting theme}
\StringTok{  }
\StringTok{  }\KeywordTok{aes}\NormalTok{(}\DataTypeTok{x      =}\NormalTok{ Cement,                       }\CommentTok{# x-value}
      \DataTypeTok{y      =}\NormalTok{ Compressive_Strength_28dy) }\OperatorTok{+}\StringTok{  }\CommentTok{# y-value}

\StringTok{  }\KeywordTok{ggtitle}\NormalTok{(}\StringTok{"Yeh Superplasticizer Tests"}\NormalTok{) }\OperatorTok{+}\StringTok{    }\CommentTok{# Custom Title}
\StringTok{  }
\StringTok{  }\KeywordTok{xlab}\NormalTok{(}\KeywordTok{expression}\NormalTok{(}\StringTok{'Cement Amount (kg m'}\OperatorTok{^-}\DecValTok{3}\OperatorTok{*}\StringTok{")"}\NormalTok{)) }\OperatorTok{+}\StringTok{   }\CommentTok{# x-label}
\StringTok{  }\KeywordTok{ylab}\NormalTok{(}\StringTok{"28-dy Compressive Strength (MPa)"}\NormalTok{)      }\OperatorTok{+}\StringTok{   }\CommentTok{# y-label}

\StringTok{  }\KeywordTok{geom_point}\NormalTok{(}\DataTypeTok{colour=}\StringTok{"darkgrey"}\NormalTok{) }\OperatorTok{+}\StringTok{  }\CommentTok{# plot points}
\StringTok{  }\KeywordTok{geom_smooth}\NormalTok{(}\DataTypeTok{method  =} \StringTok{"lm"}\NormalTok{,    }\CommentTok{# use a simple linar model}
              \DataTypeTok{formula =}\NormalTok{ y }\OperatorTok{~}\StringTok{ }\NormalTok{x,   }\CommentTok{# lm-style formula}
              \DataTypeTok{se      =} \OtherTok{TRUE}\NormalTok{,    }\CommentTok{# splay Confidence Intervals}
              \DataTypeTok{level   =} \FloatTok{0.95}\NormalTok{,    }\CommentTok{# Confidene Level to Map Out}
              \DataTypeTok{colour  =} \StringTok{"black"}\NormalTok{, }\CommentTok{# regression line color}
              \DataTypeTok{size    =} \FloatTok{0.5}\NormalTok{)     }\CommentTok{# line thickness}
\end{Highlighting}
\end{Shaded}

\includegraphics{Concrete_Multivariate_Regression_Using_R_with_Tidyverse_files/figure-latex/unnamed-chunk-33-1.pdf}

The line here looks like a positive correlation between the cement
amount and the resulting strength.

Let's try water:

\begin{Shaded}
\begin{Highlighting}[]
\CommentTok{# getting the linear model}


\NormalTok{linear_model.S_v_w =}\StringTok{  }\KeywordTok{lm}\NormalTok{(}\DataTypeTok{formula =}\NormalTok{ Compressive_Strength_28dy }\OperatorTok{~}\StringTok{ }\NormalTok{Water, }\CommentTok{# your formula y ~ x}
                         \DataTypeTok{data    =}\NormalTok{ concrete   )                           }\CommentTok{# the data frame}

\KeywordTok{summary}\NormalTok{(linear_model.S_v_w)}
\end{Highlighting}
\end{Shaded}

\begin{verbatim}
## 
## Call:
## lm(formula = Compressive_Strength_28dy ~ Water, data = concrete)
## 
## Residuals:
##     Min      1Q  Median      3Q     Max 
## -19.359  -5.451  -0.986   4.690  18.825 
## 
## Coefficients:
##             Estimate Std. Error t value Pr(>|t|)    
## (Intercept)  55.4824     7.3978    7.50  2.5e-11 ***
## Water        -0.0986     0.0373   -2.64   0.0096 ** 
## ---
## Signif. codes:  0 '***' 0.001 '**' 0.01 '*' 0.05 '.' 0.1 ' ' 1
## 
## Residual standard error: 7.6 on 101 degrees of freedom
## Multiple R-squared:  0.0646, Adjusted R-squared:  0.0554 
## F-statistic: 6.98 on 1 and 101 DF,  p-value: 0.00956
\end{verbatim}

\begin{Shaded}
\begin{Highlighting}[]
\CommentTok{# Making a simple X-Y scatterplot and adding a regression to it}

\KeywordTok{ggplot}\NormalTok{(}\DataTypeTok{data =}\NormalTok{ concrete) }\OperatorTok{+}\StringTok{                }\CommentTok{# invoke graphics environment using a given dataframe}
\StringTok{  }
\StringTok{  }\KeywordTok{theme_bw}\NormalTok{( ) }\OperatorTok{+}\StringTok{                           }\CommentTok{# changing the plotting theme}
\StringTok{  }
\StringTok{  }\KeywordTok{aes}\NormalTok{(}\DataTypeTok{x      =}\NormalTok{ Water,                      }\CommentTok{# x-value}
      \DataTypeTok{y      =}\NormalTok{ Compressive_Strength_28dy) }\OperatorTok{+}\StringTok{  }\CommentTok{# y-value}

\StringTok{  }\KeywordTok{ggtitle}\NormalTok{(}\StringTok{"Yeh Superplasticizer Tests"}\NormalTok{) }\OperatorTok{+}\StringTok{    }\CommentTok{# Custom Title}
\StringTok{  }
\StringTok{  }\KeywordTok{xlab}\NormalTok{(}\KeywordTok{expression}\NormalTok{(}\StringTok{'Water Amount (kg m'}\OperatorTok{^-}\DecValTok{3}\OperatorTok{*}\StringTok{")"}\NormalTok{)) }\OperatorTok{+}\StringTok{  }\CommentTok{# x-label}
\StringTok{  }\KeywordTok{ylab}\NormalTok{(}\StringTok{"28-dy Compressive Strength (MPa)"}\NormalTok{)      }\OperatorTok{+}\StringTok{   }\CommentTok{# y-label}

\StringTok{  }\KeywordTok{geom_point}\NormalTok{(}\DataTypeTok{colour=}\StringTok{"darkblue"}\NormalTok{) }\OperatorTok{+}\StringTok{  }\CommentTok{# plot points}
\StringTok{  }
\StringTok{  }\KeywordTok{geom_smooth}\NormalTok{(}\DataTypeTok{method  =} \StringTok{"lm"}\NormalTok{,    }\CommentTok{# use a simple linar model}
              \DataTypeTok{formula =}\NormalTok{ y }\OperatorTok{~}\StringTok{ }\NormalTok{x,   }\CommentTok{# lm-style formula}
              \DataTypeTok{se      =} \OtherTok{TRUE}\NormalTok{,    }\CommentTok{# splay Confidence Intervals}
              \DataTypeTok{level   =} \FloatTok{0.95}\NormalTok{,    }\CommentTok{# Confidene Level to Map Out}
              \DataTypeTok{colour  =} \StringTok{"blue"}\NormalTok{,  }\CommentTok{# regression line color}
              \DataTypeTok{fill    =} \StringTok{"cyan"}\NormalTok{,  }\CommentTok{# NEW: fill for confidence limits}
              \DataTypeTok{size    =} \FloatTok{0.5}\NormalTok{)     }\CommentTok{# line thickness}
\end{Highlighting}
\end{Shaded}

\includegraphics{Concrete_Multivariate_Regression_Using_R_with_Tidyverse_files/figure-latex/unnamed-chunk-35-1.pdf}

Looking up back the tables none of the variables

\hypertarget{multivariate-linear-regression}{%
\section{7. Multivariate Linear
Regression}\label{multivariate-linear-regression}}

And now we're going to do something about that!

We're now going to use not just one independent variable\ldots{} but all
7 of them!

The good news is that it follows the same form as the simple linear
regression. This time we string along all of our independent variables
with in our formula prototype.

Our formula now has multiple independent values but still follows the
same style of solution\ldots{}

\(\widehat{y}(\mathbf{x}) = {\alpha_0}+{\alpha_1} x_1 + {\alpha_2} x_2 + {\alpha_2} x_3 + ... +{\alpha_n} x_n\)

\begin{Shaded}
\begin{Highlighting}[]
\NormalTok{linear_model.S_v_all <-}\StringTok{ }\KeywordTok{lm}\NormalTok{(}\DataTypeTok{data    =}\NormalTok{ concrete,                             }\CommentTok{# your data frame}
                           \DataTypeTok{formula =}\NormalTok{ Compressive_Strength_28dy }\OperatorTok{~}\StringTok{ }\NormalTok{Cement }\OperatorTok{+}\StringTok{  }\CommentTok{# your formula}
\StringTok{                                                                 }\NormalTok{Slag }\OperatorTok{+}
\StringTok{                                                                 }\NormalTok{Fly_Ash }\OperatorTok{+}
\StringTok{                                                                 }\NormalTok{Water }\OperatorTok{+}
\StringTok{                                                                 }\NormalTok{Superplasticizer }\OperatorTok{+}
\StringTok{                                                                 }\NormalTok{Fine_Aggregates }\OperatorTok{+}
\StringTok{                                                                 }\NormalTok{Coarse_Aggregates)  }
\end{Highlighting}
\end{Shaded}

And here are these results\ldots{}

\begin{Shaded}
\begin{Highlighting}[]
\KeywordTok{summary}\NormalTok{(}\DataTypeTok{object =}\NormalTok{ linear_model.S_v_all)}
\end{Highlighting}
\end{Shaded}

\begin{verbatim}
## 
## Call:
## lm(formula = Compressive_Strength_28dy ~ Cement + Slag + Fly_Ash + 
##     Water + Superplasticizer + Fine_Aggregates + Coarse_Aggregates, 
##     data = concrete)
## 
## Residuals:
##    Min     1Q Median     3Q    Max 
## -5.841 -1.706 -0.283  1.299  7.942 
## 
## Coefficients:
##                   Estimate Std. Error t value Pr(>|t|)   
## (Intercept)       139.7815    71.1013    1.97   0.0522 . 
## Cement              0.0614     0.0228    2.69   0.0084 **
## Slag               -0.0297     0.0318   -0.94   0.3520   
## Fly_Ash             0.0505     0.0232    2.18   0.0316 * 
## Water              -0.2327     0.0717   -3.25   0.0016 **
## Superplasticizer    0.1031     0.1346    0.77   0.4453   
## Fine_Aggregates    -0.0391     0.0288   -1.36   0.1783   
## Coarse_Aggregates  -0.0556     0.0274   -2.03   0.0455 * 
## ---
## Signif. codes:  0 '***' 0.001 '**' 0.01 '*' 0.05 '.' 0.1 ' ' 1
## 
## Residual standard error: 2.6 on 95 degrees of freedom
## Multiple R-squared:  0.897,  Adjusted R-squared:  0.889 
## F-statistic:  118 on 7 and 95 DF,  p-value: <2e-16
\end{verbatim}

Our regression coefficients are still here under the ``Estimate'' column
as are our Standard Error of our Estimate and our Coeff of
Determination.

Also we can now take a good look at those asterisks at the end of line
with the parameter coefficients. These can explain which independent
variables do the heaviest lifting in our regression. The more asterisks,
the more important the dependent variable is to the larger multivariate
regression. Here, we can see that the Cement and Water are doing most of
the ``work'' in fitting our suite of independent variables to our
dependent variable of Compressive Strength.

Finally there is the P parameter for which the smaller it is, the better
we can say that the relationship that we've made with our regression
represents our dependent variable.

Now\ldots{} on to looking at our results.

Here is where viewing the results of the regression is tricky.

We have 7 independent variables but we'd like to see the impact of the
fit if all 7 variables on our strength

When I do this I like to plot the true y value against my regression
y(x1,x2,x3,..)

So to do this I will take the fitted values of y and plot them against
the original values of y

Getting the fitted values is easy.

I'm using the get\_regression\_points function which adds the modeled
``y-hat'' value to the dataframe of all of the other values
\href{https://www.rdocumentation.org/packages/stats/versions/3.5.1/topics/fitted}{get\_regression\_points()}
function.

The fitted version is the dependent variable w/ a "\_hat"" at the end

\begin{Shaded}
\begin{Highlighting}[]
\NormalTok{fitted.S_v_all =}\StringTok{ }\KeywordTok{get_regression_points}\NormalTok{(}\DataTypeTok{model =}\NormalTok{ linear_model.S_v_all)}

\KeywordTok{print}\NormalTok{(fitted.S_v_all)}
\end{Highlighting}
\end{Shaded}

\begin{verbatim}
## # A tibble: 103 x 11
##       ID Compressive_Str~ Cement  Slag Fly_Ash Water Superplasticizer
##    <int>            <dbl>  <dbl> <dbl>   <dbl> <dbl>            <dbl>
##  1     1             35.0    273    82     105   210                9
##  2     2             41.1    163   149     191   180               12
##  3     3             41.8    162   148     191   179               16
##  4     4             42.1    162   148     190   179               19
##  5     5             26.8    154   112     144   220               10
##  6     6             25.2    147    89     115   202                9
##  7     7             38.9    152   139     178   168               18
##  8     8             36.6    145     0     227   240                6
##  9     9             32.7    152     0     237   204                6
## 10    10             38.5    304     0     140   214                6
## # ... with 93 more rows, and 4 more variables: Fine_Aggregates <dbl>,
## #   Coarse_Aggregates <dbl>, Compressive_Strength_28dy_hat <dbl>,
## #   residual <dbl>
\end{verbatim}

And finally we can plot our actual vs modeled values. (I'm adding a
trend line)

\begin{Shaded}
\begin{Highlighting}[]
\CommentTok{# Making a simple X-Y scatterplot and adding a regression to it}

\KeywordTok{ggplot}\NormalTok{(}\DataTypeTok{data =}\NormalTok{ fitted.S_v_all) }\OperatorTok{+}\StringTok{           }\CommentTok{# invoke graphics environment using a given dataframe}
\StringTok{  }
\StringTok{  }\KeywordTok{theme_bw}\NormalTok{( ) }\OperatorTok{+}\StringTok{                           }\CommentTok{# changing the plotting theme}
\StringTok{  }
\StringTok{  }\KeywordTok{aes}\NormalTok{(}\DataTypeTok{x      =}\NormalTok{ Compressive_Strength_28dy,    }\CommentTok{# x-value}
      \DataTypeTok{y      =}\NormalTok{ Compressive_Strength_28dy_hat) }\OperatorTok{+}\StringTok{  }\CommentTok{# y-value}

\StringTok{  }\KeywordTok{ggtitle}\NormalTok{(}\StringTok{"Yeh Superplasticizer Tests"}\NormalTok{,}
          \DataTypeTok{subtitle =} \StringTok{"28-dy Compressive Strength (MPa)"}\NormalTok{) }\OperatorTok{+}\StringTok{    }\CommentTok{# EDITED: Custom Title now with a subtitle}
\StringTok{  }
\StringTok{  }\KeywordTok{ylab}\NormalTok{(}\StringTok{"Modelled"}\NormalTok{)     }\OperatorTok{+}\StringTok{ }\CommentTok{# y-label}
\StringTok{  }\KeywordTok{xlab}\NormalTok{(}\StringTok{"Observed"}\NormalTok{)     }\OperatorTok{+}\StringTok{ }\CommentTok{# x-label}

\StringTok{  }\KeywordTok{geom_point}\NormalTok{(}\DataTypeTok{colour=}\StringTok{"darkred"}\NormalTok{) }\OperatorTok{+}\StringTok{  }\CommentTok{# plot points}
\StringTok{  }
\StringTok{  }\KeywordTok{geom_smooth}\NormalTok{(}\DataTypeTok{method  =} \StringTok{"lm"}\NormalTok{,      }\CommentTok{# use a simple linar model}
              \DataTypeTok{formula =}\NormalTok{ y }\OperatorTok{~}\StringTok{ }\NormalTok{x,     }\CommentTok{# lm-style formula}
              \DataTypeTok{se      =} \OtherTok{TRUE}\NormalTok{,      }\CommentTok{# display Confidence Intervals}
              \DataTypeTok{level   =} \FloatTok{0.95}\NormalTok{,      }\CommentTok{# Confidene Level to Map Out}
              \DataTypeTok{colour  =} \StringTok{"red"}\NormalTok{,     }\CommentTok{# regression line color}
              \DataTypeTok{fill    =} \StringTok{"magenta"}\NormalTok{, }\CommentTok{# fill for confidence limits}
              \DataTypeTok{size    =} \FloatTok{0.5}\NormalTok{)  }\OperatorTok{+}\StringTok{    }\CommentTok{# line thickness}
\StringTok{  }
\StringTok{  }\KeywordTok{geom_abline}\NormalTok{(}\DataTypeTok{slope     =} \DecValTok{1}\NormalTok{,       }\CommentTok{# NEW: add a very simple line}
              \DataTypeTok{intercept =} \DecValTok{0}\NormalTok{,       }\CommentTok{#  (for a 1:1 reference)}
              \DataTypeTok{color     =} \StringTok{"grey"}\NormalTok{,}
              \DataTypeTok{linetype  =} \StringTok{"dashed"}\NormalTok{) }\OperatorTok{+}

\StringTok{  }\KeywordTok{coord_fixed}\NormalTok{(}\DataTypeTok{ratio =} \DecValTok{1}\NormalTok{)           }\CommentTok{# NEW: make the aspect ratio }
\end{Highlighting}
\end{Shaded}

\includegraphics{Concrete_Multivariate_Regression_Using_R_with_Tidyverse_files/figure-latex/unnamed-chunk-39-1.pdf}

\begin{Shaded}
\begin{Highlighting}[]
                                   \CommentTok{#   (I like my plots square)}
\end{Highlighting}
\end{Shaded}

And here we have a nice plot showing our true vs predicted values.

\hypertarget{regression-quality-metrics}{%
\section{8. Regression Quality
Metrics}\label{regression-quality-metrics}}

And to close things off, we can do some general error metrics that may
be useful..

First, the Mean Squared Error (MSE) or Bias\ldots{} (if we are too high
or too low)

\(BIAS = MSE = \frac{1}{N} \sum_{i=1}^{n} (\widehat{y}(\overrightarrow{x_i})-y_i) = \overline{(\widehat{y}(\overrightarrow{x_i})-y_i)}\)

\begin{Shaded}
\begin{Highlighting}[]
  \CommentTok{# Calculate Bias (MSE)}

\NormalTok{  bias =}\StringTok{ }\KeywordTok{mean}\NormalTok{(fitted.S_v_all}\OperatorTok{$}\NormalTok{Compressive_Strength_28dy_hat }\OperatorTok{-}\StringTok{ }
\StringTok{                 }\NormalTok{fitted.S_v_all}\OperatorTok{$}\NormalTok{Compressive_Strength_28dy)}
  
  \KeywordTok{print}\NormalTok{(}\KeywordTok{str_c}\NormalTok{(}\StringTok{" Mean Squared Error (MSE) or Bias: "}\NormalTok{, bias))}
\end{Highlighting}
\end{Shaded}

\begin{verbatim}
## [1] " Mean Squared Error (MSE) or Bias: 2.91262135922341e-05"
\end{verbatim}

For a linear or multivariate regression the average of our residuals
(the difference between each observation and prediction) \emph{should}
be zero.

The root mean squared error (RMSE) is shown here. It shouldn't be zero
since the residuals are squared before summing them up. We technically
should use the standard error of the estimate, but RMSE remains a common
error metric. We can always do both. The standard error of the estimate
takes into account the degrees of freedom which which now includes all
of the independent variables (p). We can get the standard error of the
estimate from our

\(RMSE = \sqrt{ \frac{1}{N} \sum_{i=1}^{n} (\widehat{y}(\overrightarrow{x_i})-y_i)^2 } = \sqrt{\overline{(\widehat{y}(\overrightarrow{x_i})-y_i)^2} }\)

\(s_{e}\) or
\(s_{y/x} = \sqrt{ \frac{1}{N-p-1} \sum_{i=1}^{n} (\widehat{y}(\overrightarrow{x_i})-y_i)^2 }\)

\begin{Shaded}
\begin{Highlighting}[]
  \CommentTok{# Calculate RMSE}

\NormalTok{  rmse =}\StringTok{ }\KeywordTok{sqrt}\NormalTok{(}\KeywordTok{mean}\NormalTok{( (fitted.S_v_all}\OperatorTok{$}\NormalTok{Compressive_Strength_28dy_hat }\OperatorTok{-}
\StringTok{                       }\NormalTok{fitted.S_v_all}\OperatorTok{$}\NormalTok{Compressive_Strength_28dy)}\OperatorTok{^}\DecValTok{2}\NormalTok{)  )}
  
  \KeywordTok{print}\NormalTok{(}\KeywordTok{str_c}\NormalTok{(}\StringTok{"     Root Mean Squared Error (RMSE): "}\NormalTok{,  }
\NormalTok{              rmse))}
\end{Highlighting}
\end{Shaded}

\begin{verbatim}
## [1] "     Root Mean Squared Error (RMSE): 2.50527978593714"
\end{verbatim}

\begin{Shaded}
\begin{Highlighting}[]
  \KeywordTok{print}\NormalTok{(}\KeywordTok{str_c}\NormalTok{(}\StringTok{"Standard Error of the Estimate (se): "}\NormalTok{, }
              \KeywordTok{summary}\NormalTok{(linear_model.S_v_all)}\OperatorTok{$}\NormalTok{sigma))  }\CommentTok{# you have to dig for this one!}
\end{Highlighting}
\end{Shaded}

\begin{verbatim}
## [1] "Standard Error of the Estimate (se): 2.60865763395229"
\end{verbatim}

And finally our correlation coefficient (which is basically our
coefficient of determination before the ``R'' is ``squared'')

\begin{Shaded}
\begin{Highlighting}[]
  \CommentTok{# Get The Unadjusted Correlation Coefficient}

\NormalTok{  r =}\StringTok{ }\KeywordTok{cor}\NormalTok{(}\DataTypeTok{x =}\NormalTok{ fitted.S_v_all}\OperatorTok{$}\NormalTok{Compressive_Strength_28dy,     }\CommentTok{# the x-value }
          \DataTypeTok{y =}\NormalTok{ fitted.S_v_all}\OperatorTok{$}\NormalTok{Compressive_Strength_28dy_hat, }\CommentTok{# the y-value}
          \DataTypeTok{method =} \StringTok{"pearson"}                                \CommentTok{# method of correlation}
\NormalTok{          )}
  
  \KeywordTok{print}\NormalTok{(}\KeywordTok{str_c}\NormalTok{(}\StringTok{"                        correlation coefficient (r): "}\NormalTok{, r))}
\end{Highlighting}
\end{Shaded}

\begin{verbatim}
## [1] "                        correlation coefficient (r): 0.94701611900088"
\end{verbatim}

\begin{Shaded}
\begin{Highlighting}[]
  \KeywordTok{print}\NormalTok{(}\KeywordTok{str_c}\NormalTok{(}\StringTok{"                  coefficient of determination (r²): "}\NormalTok{, r}\OperatorTok{^}\DecValTok{2}\NormalTok{, }
                                                                 \StringTok{" "}\NormalTok{, }
                                 \KeywordTok{summary}\NormalTok{(linear_model.S_v_all)}\OperatorTok{$}\NormalTok{r.squared))}
\end{Highlighting}
\end{Shaded}

\begin{verbatim}
## [1] "                  coefficient of determination (r²): 0.896839529647489 0.896837609814009"
\end{verbatim}

\begin{Shaded}
\begin{Highlighting}[]
  \KeywordTok{print}\NormalTok{(}\KeywordTok{str_c}\NormalTok{(}\StringTok{"adjusted coefficient of determination (Adjusted r²): "}\NormalTok{, }
               \KeywordTok{summary}\NormalTok{(linear_model.S_v_all)}\OperatorTok{$}\NormalTok{adj.r.squared))}
\end{Highlighting}
\end{Shaded}

\begin{verbatim}
## [1] "adjusted coefficient of determination (Adjusted r²): 0.889236170537147"
\end{verbatim}

And with that, we're done\ldots{} Once again, this exercise demonstrates
a lot of tricks just to show how you can use R for various statistics.
You may not use all of them in your encouters with R or even at all, but
you may be able to cannibalize some of the tricks here for other
applications.


\end{document}
